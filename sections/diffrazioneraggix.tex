%%
%% Author: dariochinelli
%% 2021-03-15
%%


\section{Diffrazione raggi X}
Nel 1913 Bragg scoprì che i solidi cristallini producevano pattern molto particolari nella diffrazione di raggi $X$.
Scoprì infatti che questi cristalli, a determinate lunghezze d'onda, producono picchi di intensità di radiazione diffusa ad angoli ben precisi.

\begin{figure}[h]
\centering
\includegraphics[scale=0.5]{/diffusione_raggiX}
\caption{Picchi di intensità per un materiale cristallino con struttura cubica}
\end{figure}

Cos'è un \textit{reticolo cristallino}? 
Si consideri ad esempio un pezzo di ferro, o alluminio, essi possiedono un \textit{ordine cristallino}, ovvero gli atomi che li compongono sono localizzati nello spazio di una struttura ordinata.
È come avere una matrice di atomi, posti in posizioni ben precise e ripetute "infinitamente" che compongono il materiale.
Ebbene, per semplicità, consideriamo celle di forma cubica.

Per angoli ben precisi vedo quindi picchi della radiazione X, ciò è dovuto all'interferenza costruttiva che si ottiene dalla differenza di cammino ottico fra i due raggi riflessi da due atomi della struttura cristallina.
Venne spiegato da Max Von Laue.
\begin{figure}[h]
\centering
\includegraphics[scale=0.6]{/atomi_radiazineX}
\caption{Diagramma esplicativo della situazione}
\end{figure}

Pensando ad un reticolo di diffrazione: ogni fenditura è sorgente di onde;
Allo stesso modo ogni atomo si comporta come una sorgente d'onde e si verifica l'interferenza costruttiva.

Il contributo fondamentale per capire il fenomeno fu dato da William Henry Bragg e da William Lawrence Bragg, padre e figlio, i quali conoscendo i lavori di Von Laue capirono che si poteva spiegare il fenomeno assumendo che: \\ 
\textit{"ogni raggio diffratto esiste un set di piani reticolari cosicché il raggio diffratto appare come riflesso specularmente da tale set di piani"}.
L'ipotesi di Bragg è che i piani siano semi-riflettenti.
\begin{figure}[h]
\centering
\includegraphics[scale=0.7]{/schema_bragg}
\caption{Schema cammini ottici}
\label{cammino_ottico}
\end{figure}

Come si vede in figura \ref{cammino_ottico} la differenza di cammino ottico sarà data da
\begin{equation}
\begin{split}
& \Delta = 2 (d \sin\theta) \\ 
& 2 d \sin\theta = n \lambda 
\end{split}
\end{equation}

Dove l'ultima equazione è la parte analitica della \textbf{Legge di Bragg}.

Bragg interpreta il fenomeno come una riflessione e non come una diffrazione, quando è verificata la condizione precedente.
Significa quindi che si tratta di picchi di \textit{riflessione} detti "picchi di riflessione di Bragg".
Per ogni tipo di cristallo potrò osservare tanti picchi di diffrazione ad angoli diversi, che corrispondono ad una riflessione da un set reticolare diverso.

Questo fenomeno è utile per investigare la materia:
sottoponendo un campione ad un fascio di raggi X risalgo alla sua struttura cristallina attraverso l'osservazione del pattern di diffrazione ottenuto.

\begin{figure}[h]
\centering
\includegraphics[scale=0.5]{/pattern_diffrazione_X}
\caption{Esempio di spettro di diffrazione}
\end{figure}

