%%
%% Author: dariochinelli
%% 2021-1-23
%%

\section{Gas di elettroni}

Gli elettroni sono fermioni, la loro statistica è perciò descritta dal modello di Fermi-Dirac, poiché si tratta di particelle soggette al principio di esclusione di Pauli.

Sostituisco nella formula di Fermi-Dirac

\begin{equation}
\frac{n_s}{g_s} = \frac{1}{e^{\alpha + \beta E_s} + 1 } \quad \mbox{con} \quad
	\begin{cases}
		\alpha = - \frac{ E_F}{k T } \quad \mbox{con $E_F$ funzione della temperatura} \\
		\beta = \frac{ 1}{k T }
	\end{cases} \\
\end{equation}

\begin{equation}
\frac{n_s}{g_s} = \frac{1}{e^{\alpha + \beta E_s} + 1 }
= \frac{ 1}{e^{ \frac{ E_s - E_F}{k T } } +1 }
\end{equation}

Quando $E_s = E_F$ si ha che $\frac{ n_s}{g_s } = \frac{ 1}{2 }$ \\
e ciò è particolarmente evidente per $T \not = 0$, infatti $\lim_{T \to 0} \frac{ n_s}{g_s } = 0$

Si definisce $E_F(T)$ l'\underline{energia di Fermi}, e si parla di \underline{livello di Fermi} quando si ha $E_F(0)$. Valutiamo ora lo studio di funzione di $\frac{ n_s}{g_s }$ al variare di $T$. \\

Analizziamo tre casi significativi dell'andamento del rapporto $\frac{ n_s}{g_s }$ rispetto a diverse temperature:

\begin{figure}[h]
\centering
\includegraphics[scale=0.13]{/Gas_elettroni_temperature}
\caption{Grafici casi limite temperature gas di elettroni}
\end{figure}

\begin{enumerate}[label=(\alph*)]
\item Caso $T = 0$ quindi $\alpha = - \infty$,  gli stati di più bassa energia vengono occupati per primi, gli altri a seguire. 
In accordo con il Principio di esclusione di Pauli.
\item Caso $T \not = 0$  con $T \ll T_F$ \quad $\Bigl( dove T_F = \frac{ E_F(0)}{K }   \Bigr)$, si può notare come la funzione a gradino si modifichi lievemente,
il disegno non è in scala ma rende l'idea, è noto che $E_F(0) \gg kT$.
Siamo infatti in una situazione per cui la differenza dal caso precedente è minima, poiché in tale condizione $E_F(T) \sim E_F(0)$.
L'energia termica del sistema è usata per promuovere i fermioni dagli stati di energia un po' sotto $E_F$ agli stati un po' sopra $E_F$, 
in particolare la variazione di energia è ristretta ad una regione (del grafico) di lunghezza $kT$, che è l'energia di ogni particella.
\item Ad alte temperature abbiamo $T \gg T_F$ il numero di particelle per stato quantistico per la distribuzione classica è più piccolo di 1,
le distribuzioni quantistiche si fondono con quella classica.
\end{enumerate}

Calcoliamo ora il numero di particelle: dalla statistica di Fermi-Dirac si ha che

\begin{gather*} 
n(E)dE = \frac{ g(E)}{e^{ \frac{ E - E_F}{kT } } + 1 } \\
\mbox{dove} \quad g(E)= \frac{4 \pi V (2m^3)^{\frac{1}{2}} E^{\frac{1}{2}}}{h^3} \\
\mbox{spin } \Bigl(  \pm \frac{ 1}{2 }  \Bigr) \Rightarrow \quad g(E)= \frac{8 \pi V (2m^3)^{\frac{1}{2}} E^{\frac{1}{2}}}{h^3}
\end{gather*}

trattandosi di elettroni occorre moltiplicare per due a causa della doppia possibilità dello spin

\begin{equation}
N = \int_0^{\infty} n(E)dE = \int_0^{\infty} \frac{8 \pi V (2m^3)^{\frac{1}{2}} E^{\frac{1}{2}}}{h^3} \frac{ dE}{e^{ \frac{ E-E_F}{kT } } + 1 }
\end{equation}

Notiamo che allo zero assoluto si ha che 

\begin{equation}
\frac{ 1 }{e^{ \frac{ E - E_F}{kT } } + 1 } = 1 \quad
\mbox{come una funzione a gradino: } \quad
\begin{cases}
	1 \quad 0 \le E \le E_F \\
	0 \quad > E_F
\end{cases}
\end{equation}
dunque si semplifica scrivendo che 

\begin{equation}
\begin{split}
N = \int_0^{E_F(0)} g(E)dE & =  \int_0^{E_F(0)} \frac{8 \pi V (2m^3)^{\frac{1}{2}} E^{\frac{1}{2}}}{h^3} dE \\
& = \frac{16 \pi V (2m^3)^{\frac{1}{2}} }{3 h^3} [E_F(0)]^{\frac{3}{2}}
\end{split}
\end{equation}

\begin{equation}
\Longrightarrow \quad E_F(0) = \frac{ h^2}{8m } \Bigl(  \frac{ 3}{\pi } \frac{ N}{V }  \Bigr)^{ \frac{ 3}{ 2} } = \frac{ h^2}{8m } \Bigl(  \frac{ 3}{\pi } n \Bigr)^{ \frac{ 3}{ 2} }
\end{equation}

dove $n = \frac{ N}{V }$.
Da un punto di vista matematico si è calcolato l'area del sottografico come in figura

\begin{figure}[h]
\centering
\includegraphics[scale=0.07]{/gas_elettroni_areagrafico}
\caption{area grafico}
\end{figure}

Ad ogni modo si vede che il valore del livello di Fermi $E_F(0)$ aumenta all'aumentare del numero di particelle presenti nel sistema.
Calcoliamo quindi l'energia:

\begin{equation}
\begin{split}
U & = \int_0^{E_F(0)} E g(E)dE = \frac{8 \pi V (2m^3)^{\frac{1}{2}} }{h^3} \int_0^{E_F(0)} E^{ \frac{ 3}{2 } } dE \\
& = \frac{ 16}{5 } = \frac{ \pi V (2m)^{ \frac{ 1}{2 } }}{h^{ 3 } } \Bigl[ E_F(0) \Bigr]\frac{ 5}{2} \\
& = \frac{ 3}{2} N E_F(0)
\end{split}
\end{equation}


Dunque $\frac{ U}{N} = \frac{ 3}{5} E_F(0)$ e ciò significa che ogni particella, anche allo zero assoluto, 
possiede almeno un'energia minima pari a $\frac{ 3}{5} E_F(0)$, quindi anche per $T = 0 K $ si ha energia non nulla!
Come corollario si vede che il gas di fermioni, allo zero assoluto, possiede una pressione non nulla, detta
\underline{Pressione di Pauli}, pari a 

\begin{equation}
P = \frac{ 2}{3} \frac{ U}{N} = \frac{ 2}{5} n E_F(0)
\end{equation}

Poiché $\alpha = -\frac{ E_F}{k T} \quad \Rightarrow \quad \xi = e^{-\alpha} = e^{ \frac{ E_F}{k T}}$ allora si ha che per $T=0 \Rightarrow \xi = +\infty $
e si è in presenza di un gas totalmente degenere, che significa, nel caso di fermioni, che le particelle sono tutte "ammassate", pur rispettando il Principio di Esclusione di Pauli, nei livelli più bassi.
Un gas di fermioni continua a rimanere degenere se $T \ll T_F$ dove $T_F = \frac{ E_F(0)}{k}$, cioè se $kT \ll E_F(0)$, 
dove $E_F(T) \approx E_F(0)$ e quindi $\xi \gg 1$
In tal caso occorre usare la statistica di Fermi-Dirac.
Tuttavia se $kT \gg E_F(0) \Rightarrow T \gg T_F $, si deve (o si può?) applicare la statistica classica.

Supponiamo ora di avere un metallo di \underline{valenza 1} (la valenza esprime quanti elettroni di conduzione possono essere forniti da ogni atomo del metallo).
Supponiamo che la distanza fra gli elettroni sia $a \simeq \SI{0.1}{nm}$, allora una stima della densità elettronica è $n = \frac{ 1}{a^{ 3}} = a^{ -3}$.
Stimiamo $T_F$, perciò:

\begin{equation}
\begin{cases}
	T_F = \frac{ E_F(0)}{k} = \frac{ h^2}{8 k m} \Bigl(  \frac{ 3}{\pi}  \Bigr)^{ \frac{ 2}{3}} n^{ \frac{ 2}{3}} \simeq \frac{ h^2}{k m a^2} \simeq \frac{ 10^{ -68}}{10^{-23 } 10^{ 30} 10^{-20 }} \frac{ J \cdot s^2 \cdot K}{J \cdot Kg \cdot m^2} = 10^{5} K \\
	n^{ \frac{ 2}{3}} = \Bigl(  a^{ -3}  \Bigr)^{ \frac{ 2}{3}} = a^{ -2}
\end{cases}
\end{equation}

$T_F$ separa il regime di Fermi-Dirac da quello classico di Maxwell-Boltzmann, ed essendo dell'ordine di $10^{ -5}$, si ha che in situazioni ordinarie il sistema di elettroni è sempre degenere.
Il calcolo precedente è semplice perché si ha a che fare con lo zero assoluto, mentre le cose si complicano per $T \not = 0 $.

Studiamo ora la \underline{capacità termica} $C_V$, a volume costante: tale grandezza è legata a $T$ (temperatura assoluta).
Il contributo fondamentale alla capacità termica è dato dal reticolo del solido cristallino, ora studiamo il contributo elettronico.
Solo una frazione delle particelle vede variare la propria energia, perché siamo nel caso $T \not = o \ll T_F$, e tale variazione è:
$$ \Delta U \simeq N \frac{ k T}{E_F(0)} kT \simeq N\Bigl(  \frac{ T}{T_F}  \Bigr)kT $$

Poiché $C_V = \frac{ dU}{dT} \simeq N k \Bigl(  \frac{ T}{T_F}  \Bigr) $ usando $U = \frac{ 3}{2} N k T$ si ha che $C_V = \frac{ 3}{2} N k$ 
ed è allora necessario che $T$ sia confrontabile con $T_F \quad (T \approx T_F)$, ma ciò non accade nel caso ordinario per i fermioni,
per i quali quindi $C_V = C_V(T)$, e in particolare:
\begin{equation}
C_V = \frac{ \pi^2}{2} N k \Bigl(  \frac{ T}{T_F}  \Bigr)
\end{equation}

\begin{figure}[h]
\centering
\includegraphics[scale=0.1]{/buca_potenziale}
\caption{Buca di potenziale in un metallo}
\end{figure}

Consideriamo ora una \underline{buca di potenziale}: \\

Funzione lavoro $W_0 = h\nu_0 \Rightarrow V_0 = E_F(0) + W_0$ ed in generale si ha che $V_0$ è dell'ordine di qualche $eV$,
ad esempio l'argento ha $W_0 = \SI{4.7}{eV}$ e $ E_F(0) = \SI{5.5}{eV} $
Gli elettroni nel metallo riempiono i livelli energetici fino a $E_f(0)$. 
$W_0$ nei metalli è generalmente tale che $\SI{5}{eV} < V_0 < \SI{15}{eV} $

Questo spiega l'\underline{effetto termoionico}: è un fenomeno per cui un metallo riscaldato ad una temperatura sufficientemente alta inizia ad emettere elettroni presenti sulla sua superficie. A $T \not = 0 $ gli elettroni cominciano ad occupare gli strati più elevati e ad uscire dalla buca di potenziale.
















