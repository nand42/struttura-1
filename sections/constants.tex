%%
%% Author: dariochinelli
%% 2021-03-10
%%
\section{Varie info utili}
Riassumo qui molte delle formule ricavate e le costanti utilizzate nei paragrafi successivi.
\subsection{Formule}
\begin{equation}
\begin{split}
\mbox{Legge di Wien} & \quad \lambda_{max} T = \SI{2.898e-3}{m*K} = \mbox{ } const \\
\mbox{Legge di Stefan-Boltzmann} & \quad R_T = \sigma T^4 \\
\mbox{Densità energia Corpo Nero Raylight-Jeans} & \quad \rho_T d\nu = \frac{ 8 \pi \nu^2}{c^3 } kT d\nu \\
\mbox{Densità energia Corpo Nero Planck} & \quad  \rho_T d\nu = \frac{ 8 \pi \nu^2}{c^3 }\frac{ h\nu}{e^{ \frac{ h\nu}{kT } } - 1 } d\nu  \\
\mbox{Densità energia $\propto$ Radianza} & \quad \frac{ c}{4 } \rho_T(\nu) = R_T(\nu) \\
\mbox{Energia effetto fotoelettrico} & \quad K_{max} = h\nu - W_0 = eV_0 \\
\mbox{Variazione lunghezza d'onda effetto Compton} & \quad \Delta \lambda = \frac{ h}{m_{e} c } (1 - \cos \theta) = \lambda_c (1 - \cos \theta) \\
\mbox{Diffrazione differenza cammino ottico} & \quad \Delta = 2 d \sin \theta = n \lambda \\
\mbox{Equazione equil mecc elettrone (Bohr)} & \quad 
\begin{cases}
	F_{coul} = \frac{ 1}{4 \pi \varepsilon_0 } \frac{ Z e^2}{r^2 } = m \frac{ v^2}{r } = F_{cent} \\
	L = m v r =  n \hbar
\end{cases} \\
\mbox{Energia livelli orbitali (Bohr)} & \quad E = -\frac{ 1}{8 } \frac{ m e^4}{\varepsilon_0^2 h^2 } \frac{ Z^2}{n^2 } \\
\mbox{Regola di Wilson-Sommerfeld} & \quad \oint P_q dq = n_q h \\
\mbox{Hamiltoniana/Energia oscillatore armonico} & \quad H = E = K + V = \frac{ P_x^2}{2m } + \frac{ k x^2}{2 } \\
\mbox{Calore specifico Dulong-Petit} & \quad C_V = 3R \quad T \gg 0 \\
\mbox{Calore specifico Einstein} & \quad C_V = \frac{ 3 N_A (h\nu)^2}{kT^2 } e^{ -\beta h \nu } \quad \mbox{quasi-ovunque} \\
\mbox{Calore specifico Debye} & \quad C_V = 
	\begin{cases}
		\frac{ 12}{5 } \pi^4 R \Bigl(  \frac{ T}{\Theta }  \Bigr)^3 \quad \mbox{contrib. reticolare} \\
		\frac{ \pi^2}{2 } N k \Bigl(  \frac{ T}{T_F}  \Bigr) \quad \mbox{contrib. elettronico} 
	\end{cases} \\
\end{split}
\end{equation}

\subsection{Costanti}
\begin{equation}
\begin{split}
\mbox{Massa elettrone} & \quad m_e = \SI{9.1e-31}{Kg} \\
\mbox{Carica elettrone} & \quad e = \SI{1.602e-19}{C} \\
\mbox{Costante dielettrica vuoto} & \quad \varepsilon_0 = \SI{8.854e-12}{F/m} \\
\mbox{Costante di Planck} & \quad h = \SI{6.626e-34}{J s} \\
\mbox{Costante di Planck ridotta} & \quad \hbar = \frac{ h}{2\pi } = \SI{1.0545e-34}{J s} \\
\mbox{Energia minima idrogeno} & \quad \frac{ 1}{8 }\frac{m e^4 }{\varepsilon_0^2 h^2 } \Bigl(  \frac{ Z^2}{n^2 }  \Bigr)= \SI{-13.6e0}{eV} \\
\mbox{Costante di Stefan-Boltzmann} & \quad \sigma = \SI{5.67e-8}{W / ( m^2 K^4)} \\
\mbox{Costante di Boltzmann} & \quad k_B = k = \SI{1.3806e-23}{J / K} \\
\mbox{Numero (costante) di Avogadro} & \quad N_A = \SI{6.02214076e23}{mol^{-1}} \\
\mbox{Conversione J - eV} & \quad \SI{1 }{eV} = \SI{1.6022e-19}{J} \\
\mbox{Lunghezza d'onda di Compton} & \quad \lambda_c = \frac{ h}{m_e c } = \SI{2.4263e-12}{m} \\
\end{split}
\end{equation}

