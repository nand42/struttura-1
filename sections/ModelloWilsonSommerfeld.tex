%%
%% Author: dariochinelli
%% 2020-10-12
%%

\section{Modello di Wilson-Sommerfeld}

La regola di Wilson-Sommerfeld permette di generalizzare il concetto i quantizzazione per sistemi fissi, per cui le coordinate sono funzioni periodiche nel tempo.
Sia $q$ una coordinata, e $P_q$ il momento relativo ad essa, si consideri un integrale sul periodo allora esiste una quantità quantizzata associata:

$$ \oint P_q dq = n_q h $$

Nel caso dell'\textbf{oscillatore armonico} scriviamo l'energia, che corrisponde all'Hamiltoniana in quanto $\frac{\partial \mathcal{H}}{\partial t} = 0$, come 
$$ \mathcal{H} = E = K + V = \frac{P_x^2}{2m} + \frac{k x^2}{2} $$
$$\Rightarrow  \frac{P_x^2}{2mE} + \frac{x^2}{2 E/k} = 1$$

che rappresenta un'ellisse nello spazio delle fasi, con parametri $a$ e $b$:
$$a = \sqrt{\frac{2E}{k}} \mbox{  ;  }  b=\sqrt{2mE}$$

Risolviamo ora l'integrale sostituendo:

$$ \oint P_x dx = \pi a b = \frac{2\pi E}{\sqrt{k/m}} $$

si nota quindi che $\sqrt{k}{m} = \omega = 2\pi\nu$

$$ \frac{E}{\nu} = n_x h \Rightarrow  E_n = n_x h \nu$$

Consideriamo ora un $\textbf{elettrone in orbita circolare}$ per cui scriviamo il momento angolare

\begin{equation}
\begin{cases}
L = mvr \\
\theta \in [0, 2\pi]
\end{cases}
\Rightarrow \oint L d\theta
\end{equation}

$$ \oint_{0}^{2\pi} L d\theta = 2 \pi L = n h$$
$$ L = \frac{n h}{2 \pi} \Rightarrow L = n \hbar $$


Vediamo ora il $\textbf{modello atomico di Sommerfeld}$\\
Bohr non spiega il perché lo spettro sembri avere una struttura così fine, cioè perché alcune righe spettrali sono (???) in più componenti.
Sommerfeld elaborò una variazione del modello di Bohr per cerare di giustificare tale fenomeno.
Suggerisce infatti che le orbite elettroniche siano ellittiche e non circolari, ed impose il seguente sistema.

\begin{equation}
\begin{cases}
\oint L d\theta = n_{\theta} h \\
\oint P_r dr = n_r h
\end{cases}
\end{equation}
dove il secondo termine è assente nel caso circolare poiché $P_r=0$

\begin{equation}
\Rightarrow
\begin{cases}
n_{\theta} = 1, 2, 3, ... \\
L(\frac{a}{b} - 1) = n_r \hbar
\end{cases}
\end{equation}

Imponendo poi manualmente la stabilità nel caso di orbite ellittiche, si ottiene una terza equazione.
Combinando queste tre si ricava il sistema seguente

\begin{equation}
\begin{cases}
a = \frac{4\pi \epsilon_0 n^2 \hbar^2}{\mu Z e^2}\\
b = a \frac{n_{\theta}}{n}\\
E = - \bigl(\frac{1}{4\pi\epsilon_0}\bigr)^2 \frac{\mu Z^2 e^4}{2 n^2 \hbar^2}
\end{cases}
\end{equation}

\begin{equation}
\begin{cases}
n = n_{\theta} + n_r \\
n = 1, 2, 3, ... \mbox{ numero quantico principale} \\
n_{\theta} = 1, 2, 3, ... \mbox{ numero quantico azimutale}
\end{cases}
\end{equation}

Se $ n_{\theta} = n \Rightarrow b = a $ le orbite sono circolari.
Un concetto importante è quello di orbite degeneri, ovvero che esistono più orbite associate allo stesso numero quantico.
Nel caso di orbite ellittiche, è chiaro che l'elettrone viaggi più velocemente quando è più vicino al nucleo.
In tal caso quindi occorre considerare la presenza di effetti relativistici, e correggere la formula con:

$$ E = - \frac{\mu Z^2 e^4}{(4\pi\epsilon_0)^2 2 n^2} \frac{1}{\hbar^2} \biggl[ 1 + \frac{\alpha^2 Z^2}{n} \biggl( \frac{1}{n_{\theta}} - \frac{3}{4n}  \biggr)  \biggr] $$

Dove $\alpha$ è la \textbf{costante di struttura fine}:
$$ \alpha = \frac{1}{4\pi\epsilon_0} \frac{e^2}{\hbar c} \simeq \frac{1}{137}  $$

I livelli d'energia sono leggermente slittati in base a $\theta$, con questa correzione, eliminando di fatto la possibilità di orbite degeneri.
Tuttavia Sommerfeld vide che non tutte le transizioni erano permesse, ma solo quelle per cui
$$n_{\theta_i} - n_{\theta_f} = \pm 1 \mbox{ secondo la $\textit{regola di Selezione}$} $$

In realtà la struttura fine dello spettro non era dovuta ad effetti relativistici, nonostante queste formule funzionassero benissimo.
La vera causa è l'interazione $\textit{spin-orbita}$.
Nel 1923, Bohr espose il $\textit{principio di corrispondenza}$: le previsioni quantistiche devono corrispondere a quelle classiche quando gli $n$ divengono molto grandi.









