%%
%% Author: dariochinelli
%% 2020-10-04
%%

\section{Modello atomico di Bohr}

Visti i problemi del modello di Rutherford urgeva la necessità di costruire un nuovo modello atomico, che spiegasse il fenomeno degli spettri atomici discreti e che spiegasse la stabilità dell'atomo.
Nel 1913, Bohr propose il proprio modello atomico, che si fonda su dei postulati:

\begin{enumerate}
\item Un elettrone si muove su un'orbita circolare attorno al nucleo, soggetto a una attrazione Coulombiana.
\item Non tutte le orbite sono permesse, ma solo quelle tali che $L=n\hbar$, con $n=1, 2, 3, ... $.
\item L'elettrone, quando si muove sulle orbite permesse non emette radiazione elettromagnetica e ha energia costante.
\item Viene emessa radiazione elettromagnetica solo se un elettrone passa da un' orbita permessa ad un'altra, con frequenza data da $ \nu = 	\frac{(E_f - E_i)}{h}$
\end{enumerate}

Consideriamo ora un atomo come quello di idrogeno, per cui si ha un nucleo di massa $M$ e una carica $Z e$, e un singolo elettrone di carica $e$ e massa $m$.
Sia inoltre $m \ll M$ in modo che si possa considerare il nucleo fisso nello spazio.
Possiamo impostare una equazione di equilibrio meccanico, in modo da verificare quando e come l'attrazione Coulombiana compensi la forza centrifuga.

\begin{equation}
\begin{cases}
	F_{Coul.} = ma_{centr.} \rightarrow \frac{1}{4\pi \varepsilon_0} \frac{Z e^2}{r^2} = m \frac{v^2}{r} \\
	L = m v r = n \hbar 	\rightarrow 	v = \frac{n \hbar}{m r} 
\end{cases}
\Rightarrow \frac{1}{4\pi \varepsilon_0} \frac{Z e^2}{r^2} = \frac{m}{r} \frac{n^2 \hbar^2}{m^2 r^2} \\
\end{equation}

$$ r = 4\pi\varepsilon_0 \frac{ n^2 \hbar^2}{m Z e^2} $$
$$ v = \frac{1}{4\pi\varepsilon_0} \frac{Z e^2}{m r} $$

Quindi impiegando la quantizzazione del momento angolare si ottiene che $r \propto n^2$.
Ricavando quindi la velocità orbitale:

$$ v = \frac{n \hbar}{m r} = \frac{1}{4\pi\varepsilon_0} \frac{Z e^2}{n \hbar} $$

\begin{equation}
\mbox{se }n = 1
\Rightarrow
\begin{cases}
	r = \SI{5.3e-11}{m} \sim \SI{0.5}{\angstrom} \\
	v = \SI{2.2e6}{m/s} \sim 1\% \mbox{ di c}
\end{cases}
\end{equation}

Siamo quindi in un regime non relativistico, ma ciò non è più vero per $Z \gg 1$.
Dunque il modello di Bohr funziona solo per atomi leggeri.
Calcoliamo l'energia permessa:

\begin{equation}
\begin{cases}
	V = - \frac{Z e^2}{4 \pi \varepsilon_0 r} \\
	K = \frac{1}{2} m v^2 = \frac{Z e^2}{8 \pi \varepsilon_0 r} 
\end{cases}
\end{equation}

$$ E = K + V = + \frac{Z e^2}{8 \pi \varepsilon_0 r} - \frac{Z e^2}{4 \pi \varepsilon_0 r} = - K $$

Sostituendo $r$ in $E$ ottengo:

$$ E = - \frac{1}{2} \frac{1}{4\pi\varepsilon_0} \frac{Z e^2}{r} = - \frac{1}{2} \frac{1}{4\pi\varepsilon_0} Z e^2 \frac{1}{4\pi\varepsilon_0} \frac{m Z e^2}{n^2 \hbar^2}$$

e quindi l'espressione dell'energia E diventa:

$$ E = - \frac{1}{2} \biggl(\frac{1}{4\pi\varepsilon_0}\biggr)^2 \frac{m Z^2 e^4}{\hbar^2} \frac{1}{n^2} = - \frac{1}{32\pi^2\varepsilon_0^2} \frac{m Z^2 e^4}{\hbar^2} \frac{1}{n^2} $$

oppure anche:

$$ E =   - \frac{1}{2} \biggl(\frac{1}{4\pi\varepsilon_0}\biggr)^2 \frac{m e^4}{\hbar^2} \frac{Z^2}{n^2} \simeq - \frac{1}{2} \biggl(\frac{1}{4\pi\varepsilon_0}\biggr)^2 \frac{m e^4}{\hbar^2} \frac{Z^2}{n^2} $$

Poiché l'energia è proporzionale a $n^{-2}$, si ha che l'energia più negativa, e quindi minima, si ha per $n = 1$, che è la condizione di maggiore stabilità.

Lasciando arbitrari $Z$ ed $n$ si ottiene la relazione seguente

$$ E = - \frac{1}{2} \biggl(\frac{1}{4\pi (\SI{8.854e-12}{F/m})}\biggr)^2 \frac{(\SI{9.11e-31}{kg}) (\SI{1.602e-19}{C})^4 }{(\SI{6.626e-34}{J.s})^2} \frac{Z^2}{n^2}  $$

$$ E = - \frac{1}{2} \SI{4.359735e-18}{J} \frac{Z^2}{n^2} = - \SI{2.179868e-18}{J} \frac{Z^2}{n^2} $$

Nel caso in cui si cerchi l'energia minima per l'idrogeno, si pone $Z=1$ ed $n=1$ e si ottiene:

$$ E_{min} = \SI{-2.179868e-18}{J} = \SI{13.6}{eV} $$

Possiamo quindi calcolare la frequenza della radiazione emessa nel passaggio da $n_i$ a $n_f$. 

Utilizzando la relazione:

$$\nu = \frac{|E_f - E_i|}{h}$$ 

dove si tiene conto del segno del $\Delta E$ solo se si è interessati a sapere se la radiazione è emessa o assorbita, per la frequenza non è rilevante.

$$ \nu= \frac{E_i-E_f}{h} = \frac{1}{2} \biggl(\frac{1}{4\pi\varepsilon_0}\biggr)^2 \frac{m Z e^4}{\hbar^2} \frac{1}{h} \biggl(\frac{1}{n_f^2} - \frac{1}{n_i^2} \biggr)   $$


sostituendo $k = \lambda^{-1} = \frac{\nu}{c}$ 

$$ \Rightarrow k = \frac{E_i-E_f}{hc} = \biggl(\frac{1}{4\pi\varepsilon_0}\biggr)^2 \frac{m Z^2 e^4}{4\pi\hbar^3 c} \biggl(\frac{1}{n_f^2} - \frac{1}{n_i^2} \biggr) = R_{\infty} Z^2 \biggl(\frac{1}{n_f^2} - \frac{1}{n_i^2} \biggr) $$

con $R_{\infty} =\bigl(\frac{1}{4\pi\varepsilon_0}\bigr)^2 \frac{m e^4}{4\pi\hbar^3 c} $

Per cui con $Z=1$, come nel caso dell'idrogeno, si ottiene:

$$ R_{infty} Z^2 = R_{H} = \frac{9.109382 \cdot (1.602176)^4}{(4\pi)^3 (8.854188)^2 \cdot (1.054572)^3 \cdot 2.997925 } \frac{ 10^{-31} 10^{-76} } { 10^{-24} 10^{-102} 10^{8} }$$

$$ R_{H} = \SI{1.097370e7}{m^{-1}} $$

Ovvero si ottiene la formula di Rydberg.
Nello stato non eccitato l'elettone sta nel livello energetico che favorisce maggiore stabilità.
Quando può l'atomo torna allo stato di minima energia, si hanno numerose transizioni (???) per passare da $n>1$ a $n=1$.
Per $n \rightarrow \infty$ la distanza fra le bande dello spettro si assottiglia ed è difficile osservare la natura discreta, perché sembra essere un continuo.
Nel passaggio da $n$ minore ad $n$ maggiore viene emessa radiazione elettromagnetica avente una certa $\lambda$ che segue la legge scritta precedentemente.
È così dunque che si spiega la disputa: Balmer, Lymen, ecc. 
Esse dipendono dal numero quantico iniziale e da quello finale, con $n_i > n_f $.
Gli elettroni possono assorbire energia quantizzata $h\nu$ per ogni transizione, si spiega così l'assorbimento.
Con $n=1$, come nell'atomo di idrogeno, si osserva solo la serie di Lymen.
Quindi non tutte le linee di emissione possono essere viste come linee di assorbimento.
Come mai la costante è stata chiamata con $R_{\infty}$? \\
Ebbene, abbiamo sempre considerato che, poiché $M \gg m$, allora il nucleo stesse fermo e l'elettrone gli orbitasse attorno.
Tuttavia, a rigore, essi orbitano attorno al centro di massa del sistema, ed occorre considerare la \textit{massa ridotta}.

$$ \mu = \frac{m M}{m + M} \Rightarrow R_M =\frac{\mu}{m} R_{\infty} = \frac{M}{m + M} R_{\infty} $$
$$ \mbox{ma se  } \frac{M}{m} \rightarrow +\infty \Rightarrow  \lim_{m \gg M} R_M = R_{\infty}  $$


