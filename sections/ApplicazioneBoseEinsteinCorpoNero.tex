%%
%% Author: dariochinelli
%% 2021-04-08
%%


\section{Applicazione della statistica di Bose-Einstein}


\subsection{Gas di fotoni e corpo nero}
Oltre alla possibilità di utilizzare la statistica di Maxwell-Boltzmann per risolvere il problema del Corpo Nero, come affrontato nei primi capitoli, un altro modo di risolvere il problema è considerare che la luce, nella cavità di Corpo Nero, sia un \textit{gas di fotoni}: particelle con massa nulla e spin unitario.
Considerando i fotoni come \textit{bosoni}, i quali non sono soggetti al Principio di Esclusione di Pauli, e quindi tendono ad accumularsi nel livello energetico più basso, il livello fondamentale $E=0$, avente energia minore.
Occorre usare la statistica di Bose-Einstein, che scritta nel modo generico con $\beta = 1/ k_B T$ risulta essere
\begin{equation}
\begin{split}
& \frac{ n_s}{g_s } = \frac{ 1}{e^{ \alpha + \beta E_s } - 1 } = \frac{ 1}{e^{ \alpha} e^{ E / k_B T } - 1 } \\
\quad\Rightarrow\quad & n(E)dE = g(E) \frac{ 1}{e^{ \alpha} e^{ E / k_B T } - 1 } dE
\end{split}
\end{equation}
Nel caso di fotoni il numero di particelle nel sistema $N$ \textit{non è costante}, per cui il moltiplicatore di Lagrange associato è
\begin{equation}
\alpha = 0 \quad\Rightarrow\quad e^{\alpha} = 1 \quad\Rightarrow\quad  \xi = 1
\end{equation}
all'interno della cavità di Corpo Nero c'è uno scambio continuo fra il campo di radiazione e le pareti della cavità, per cui i fotoni vengono assorbiti ed emessi continuamente.
Dato che il parametro di degenerazione è pari a 1, non è certamente $\ll 1$ e dunque \emph{un gas di fotoni in equilibrio con la materia è sempre degenere}.
Quindi la legge di distribuzione dei fotoni è
\begin{equation}
n(E)dE = g(E) \frac{1}{e^{ E / k_B T } - 1 } dE
\end{equation}
L'energia di un fotone è pari a $E = h \nu$ la moltiplico per la distribuzione di Bose Einstein per ottenere la \textit{densità di energia}
\begin{equation}
\varrho_T(\nu) d\nu = \frac{h\nu n(\nu)d\nu}{V}
\label{dens_ener}
\end{equation}
Quindi esplicitando la distribuzione di Bose Einstein in funzione della frequenza si trova
\begin{equation}
n(\nu)d\nu = g(\nu) \frac{1}{e^{ h\nu / k_B T } - 1 } d\nu
\end{equation}
in cui $g(\nu)$ deriva dallo studio della funzione densità degli stati $g(E)$ da cui si ottiene quella in funzione della frequenza
\begin{equation}
\begin{split}
g(E) & = \frac{dN}{dE} = \frac{4 \pi V (2m^3)^{ \frac{1}{2} }}{h^3} E^{ \frac{1}{2} } \quad\quad \mbox{eq \ref{dens_stati}} \\
& E = \frac{p^2}{2m} \quad\Rightarrow\quad dN = g(p)dp = g(E)dE \\
g(p) & = g(E) \frac{dE}{dp} = \frac{4\pi V}{h^3} p^2 \\
& p = \frac{h}{\lambda} = \frac{h \nu}{c} \quad\Rightarrow\quad  g(\nu)d\nu  = g(p)dp \\
g(\nu) & = g(p) \frac{dp}{d\nu} = 2 \cdot \frac{4\pi V}{h^3} \nu^2
\end{split}
\end{equation}
il fattore 2 deriva dal considerare le due orientazioni possibili dello spin del fotone.
Utilizzando le espressioni ricavate sopra
\begin{equation}
n(\nu)d\nu = g(\nu) \frac{1}{e^{ h\nu / k_B T } - 1 } d\nu 
\quad\quad\quad\quad
g(\nu) = \frac{8\pi V}{h^3} \nu^2
\end{equation}
riscriviamo allora la funzione densità di energia
\begin{equation}
\varrho_T(\nu) d\nu = \frac{8 \pi \nu^2}{c^3} \cdot \frac{h\nu}{e^{\frac{h\nu}{k_B T}} - 1} d\nu
\end{equation}
che corrisponde alla \textit{Formula di Planck per il corpo nero}.


\subsection{Gas di fononi}
Un ragionamento analogo al precedente ci permette di ottenere una relazione per i fononi.
L'energia del fonone di tipo $\textbf{k}$ della branca $s$ è
\begin{equation}
E = \hbar \omega_s (\textbf{k})
\end{equation}
e l'energia totale sarà la prima delle seguenti
\begin{equation}
\begin{split}
E & = \sum_{\textbf{k}_s} \frac{\hbar \omega_s(\textbf{k})}{e^{ \hbar \omega_s (\textbf{k}) / k_B T } - 1} \\
E & = \sum_{\textbf{k}_s} \hbar \omega_s(\textbf{k}) \Bigl(  \frac{1}{e^{ \hbar \omega_s (\textbf{k}) / k_B T } - 1} + \frac{1}{2}  \Bigr)
\end{split}
\end{equation}
mentre la seconda corregge l'approssimazione fatta per l'energia dell'oscillatore armonico...
per cui considera anche l'\textit{energia di punto zero}.
Si trova allora il calore specifico, derivando la somma sulle branche dell'integrale sulla prima zona di Brillouin
\begin{equation}
c_V = \frac{\partial}{\partial T} \sum_{s} \int 
\frac{d \textbf{k}}{(2\pi)^3} 
\frac{\hbar \omega_s(\textbf{k})}{e^{ \hbar \omega_s (\textbf{k}) / k_B T } - 1}
\end{equation}
eseguendo questo conto per basse temperature posso assumere $\hbar \omega_s \textbf{k} \gg k_B T$
e considero i modi con $\omega_s \textbf{k}$ piccolo $(\textbf{k} \to 0)$ e $\omega = c k$, integrando su tutto lo spazio $k$,
trovo l'\textbf{approssimazione di Debye}
\begin{equation}
\begin{split}
& \mbox{con} \quad\quad x = \frac{\hbar c_s k}{k_B T} \quad\quad d\textbf{k} = k^2 dk 4\pi \\
c_V & \propto \frac{\partial}{\partial T} \Bigl(  3 T^4 \int_0^{\infty} \frac{x^3}{e^x - 1} dx  \Bigr) \approx T^3
\end{split}
\end{equation}


