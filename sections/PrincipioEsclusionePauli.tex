%%
%% Author: dariochinelli
%% 2021-03-23
%%

\section{Principio di Esclusione di Pauli}

Si consideri una scatola contenente due particelle, in Fisica Classica è sempre possibile distinguere il moto dell'una da quello dell'altra.
Ed anche dopo un certo tempo ed una certa evoluzione sarà possibile distinguerle.
In Meccanica Quantistica invece, a causa del Principio di Indeterminazione, le due particelle sono indistinguibili poiché le due funzioni d'onda si sovrappongono ed interferiscono.

Dunque occorre richiedere che tutti i risultati quanto-meccanici,
derivanti da calcoli fatti con la teoria quanto-meccanica di Schrodinger, non devono dipendere dall'assegnare di un'etichetta a particelle identiche: le particelle sono indistinguibili in meccanica quantistica.
È necessario scrivere quindi funzioni d'onda che tengano di questa condizione.

Partiamo scrivendo l'equazione di Schrodinger indipendente dal tempo per due particelle che non interagiscono tra loro, quindi si muovono indipendentemente l'una dall'altra
\begin{equation}
- \frac{\hbar^2}{2m} \Bigl(  \frac{\partial^2 \psi_T}{\partial x_1^2} + \frac{\partial^2 \psi_T}{\partial y_1^2} + \frac{\partial^2 \psi_T}{\partial z_1^2}  \Bigr) 
- \frac{\hbar^2}{2m} \Bigl(  \frac{\partial^2 \psi_T}{\partial x_2^2} + \frac{\partial^2 \psi_T}{\partial y_2^2} + \frac{\partial^2 \psi_T}{\partial z_2^2}  \Bigr) 
+ V_T \psi_T 
= E_T \psi_T
\end{equation}
nelle quali la funzione d'onda complessiva, che è il prodotto tra le funzioni d'onda delle due particelle, ed il potenziale, che è la somma tra i potenziali di ciascuna, sono definiti come
\begin{equation}
\begin{split}
\psi_T & = \psi(x_1, y_1, z_1) \psi(x_2, y_2, z_2) = \psi_\alpha(1) \psi_\beta(2) \\
V_T & = V(x_1, y_1, z_1) + V(x_2, y_2, z_2)
\end{split}
\end{equation}
Ogni soluzione richiede 3 numeri quantici per specificare la dipendenza da $x,y,z$.
Serve un ulteriore numero quantico per identificare lo spin.
Quindi per ogni particella, per ogni $\psi$, sono necessari quattro numeri quantici per definire lo stato quantico.
Introduco allora la notazione seguente, vista in precedenza, per indicare che la particella $1$ si trova nello stato quantico $\alpha$
\begin{equation}
\begin{split}
& \psi_{\alpha}(x_1,y_1,z_1) = \psi_{\alpha}(1) \\
& \psi_T(x_1, y_1, z_1,x_2, y_2, z_2) = \psi_\alpha(1) \psi_\beta(2)
\end{split}
\end{equation}

\paragraph{Calcoliamo la densità di probabilità}
Dati questi due possibili modi per scrivere la funzione d'onda totale
\begin{equation}
\begin{split}
& \psi_T = \psi_\alpha(1) \psi_\beta(2) \\
& \psi_T = \psi_\beta(1) \psi_\alpha(2)
\end{split}
\end{equation}
calcolo la probabilità come
\begin{equation}
\begin{split}
\mbox{ 1) } \psi^{\ast}_T\psi_T & = \psi_\alpha^\ast(1) \psi_\beta^\ast(2) \psi_\alpha(1) \psi_\beta(2) \\
\mbox{ 2) } \psi^{\ast}_T\psi_T & = \psi_\beta^\ast(1) \psi_\alpha^\ast(2) \psi_\beta(1) \psi_\alpha(2) 
\end{split}
\end{equation}
dato che le due particelle sono indistinguibili, potrei scambiare le etichette assegnate in precedenza senza che la densità di probabilità cambi, allora scambio le etichette 1 $\to$ 2 e 2 $\to$ 1 :
\begin{equation}
\begin{split}
\mbox{ 1) } \psi^{\ast}_T\psi_T & = \psi_\alpha^\ast(2) \psi_\beta^\ast(1) \psi_\alpha(2) \psi_\beta(1) \\
\mbox{ 2) } \psi^{\ast}_T\psi_T & = \psi_\beta^\ast(2) \psi_\alpha^\ast(1) \psi_\beta(2) \psi_\alpha(1) 
\end{split}
\end{equation}
otteniamo così un risultato diverso dal caso precedente, significa che non è questo il modo esatto per scrivere l'equazione di Schrodinger poiché non tiene conto del concetto di indistinguibilità.

È però possibile scrivere le equazioni, in modo che la densità di probabilità non cambi, e abbiamo due modi per farlo
\begin{equation}
\begin{split}
\psi_s & = \frac{1}{\sqrt{2}} \Bigl[ \psi_\alpha(1) \psi_\beta(2) + \psi_\beta(1) \psi_\alpha(2) \Bigr] \mbox{ soluzione simmetrica} \\
\psi_a & = \frac{1}{\sqrt{2}} \Bigl[ \psi_\alpha(1) \psi_\beta(2) - \psi_\beta(1) \psi_\alpha(2) \Bigr] \mbox{ soluzione antisimmetrica}
\end{split}
\label{funz_onda_sim_asim}
\end{equation}
sono autofunzioni diverse, che corrispondono alla stessa energia totale e che sono entrambe soluzioni dell'equazione di Schrodinger indipendente dal tempo.
Questo concetto di scambio delle etichette viene indicato con \textit{degenerazione di scambio}.
Vediamo cosa succede scambiando le etichette ottengo le espressioni
\begin{equation}
\begin{split}
\psi_s & = \frac{1}{\sqrt{2}} \Bigl[ \psi_\alpha(2) \psi_\beta(1) + \psi_\beta(2) \psi_\alpha(1) \Bigr] = \psi_s \\
\psi_a & = \frac{1}{\sqrt{2}} \Bigl[ \psi_\alpha(2) \psi_\beta(1) - \psi_\beta(2) \psi_\alpha(1) \Bigr] = - \psi_a
\end{split}
\end{equation}
il motivo per cui la primia viene chiamata "simmetrica" è dipende dal fatto che $\psi_s$ con le etichette scambiate equivale ancora a $\psi_s$;
il motivo per cui la seconda viene chiamata "antisimmetrica" è dipende dal fatto che $\psi_a$ con le etichette scambiate equivale a $-\psi_a$.

La densità della probabilità in entrambi i casi viene conservata:
\begin{equation}
\begin{split}
\psi^{\ast}_s\psi_s &\to \psi^{\ast}_s\psi_s \\
\psi^{\ast}_a\psi_a &\to (-1)^2 \psi^{\ast}_a\psi_a = \psi^{\ast}_a\psi_a\\
\end{split}
\end{equation}


\paragraph{Principio di esclusione di Pauli (1925)}
Pauli espresse il principio come: 
\textit{in un atomo multi-elettronico non ci può essere più di un elettrone nello stesso stato quantico}.
Successivamente, lo stesso Pauli, stabilì che il principio di esclusione da lui pronunciato rappresenta una proprietà propria degli elettroni e non dell'atomo a cui appartengono, per cui lo riformulò come: 
\textit{in un \underline{sistema} multi-elettronico non ci può essere più di un elettrone nello stesso stato quantico}.

Supponiamo allora il contrario, supponiamo di trovare due particelle nello stesso stato quantico $\alpha$:
\begin{equation}
\psi_a = \frac{1}{\sqrt{2}} \Bigl[ \psi_\alpha(1) \psi_\alpha(2) - \psi_\alpha(1) \psi_\alpha(2) \Bigr] = 0
\end{equation}
quindi se due particelle sono descritte da una funzione antisimmetrica non possono stare nello stesso stato quantico, risulta infatti zero la densità di probabilità di trovavele.

Posso allora riformulare il principio dicendo che: \textit{un sistema multi-elettronico (contenente molti elettroni) deve essere descritto da una autofunzione totale antisimmetrica}.

Il fatto che alcune particelle siano descritte da un'autofunzione simmetrica ed altre da un'autofunzione antisimmetrica è una proprietà fondamentale delle particelle, alla pari della massa, della carica o dello spin.
Si definisce quindi il \underline{carattere di simmetria}, vedi tabella \ref{tab_car_sim}.
% Please add the following required packages to your document preamble:
% \usepackage[table,xcdraw]{xcolor}
% If you use beamer only pass "xcolor=table" option, i.e. \documentclass[xcolor=table]{beamer}
\begin{table}[]
\centering
\begin{tabular}{cccc}
\hline
{\color[HTML]{000000} \textbf{Particella}} & {\color[HTML]{000000} \textbf{Simmetria}} & {\color[HTML]{000000} \textbf{Nome generico}} & {\color[HTML]{000000} \textbf{Spin (s)}} \\ \hline
{\color[HTML]{000000} Elettrone}           & {\color[HTML]{000000} antisimmetrica}     & {\color[HTML]{000000} fermione}               & {\color[HTML]{000000} 1/2}               \\
{\color[HTML]{000000} Positrone}           & {\color[HTML]{000000} antisimmetrica}     & {\color[HTML]{000000} fermione}               & {\color[HTML]{000000} 1/2}               \\
{\color[HTML]{000000} Protone}             & {\color[HTML]{000000} antisimmetrica}     & {\color[HTML]{000000} fermione}               & {\color[HTML]{000000} 1/2}               \\
{\color[HTML]{000000} Neutrone}            & {\color[HTML]{000000} antisimmetrica}     & {\color[HTML]{000000} fermione}               & {\color[HTML]{000000} 1/2}               \\
{\color[HTML]{000000} Muone}               & {\color[HTML]{000000} antisimmetrica}     & {\color[HTML]{000000} fermione}               & {\color[HTML]{000000} 1/2}               \\
{\color[HTML]{000000} Fotone}              & {\color[HTML]{000000} simmetrica}         & {\color[HTML]{000000} bosone}                 & {\color[HTML]{000000} 1}                 \\
{\color[HTML]{000000} Particella $\alpha$}    & {\color[HTML]{000000} simmetrica}         & {\color[HTML]{000000} bosone}                 & {\color[HTML]{000000} 0}                 \\
{\color[HTML]{000000} Atomo di elio}       & {\color[HTML]{000000} simmetrica}         & {\color[HTML]{000000} bosone}                 & {\color[HTML]{000000} 0}                 \\
{\color[HTML]{000000} Mesone $\pi$}           & {\color[HTML]{000000} simmetrica}         & {\color[HTML]{000000} bosone}                 & {\color[HTML]{000000} 0}                 \\
{\color[HTML]{000000} Deuterone}           & {\color[HTML]{000000} simmetrica}         & {\color[HTML]{000000} bosone}                 & {\color[HTML]{000000} 1}                 \\ \hline
\end{tabular}
\caption{Carattere di simmetria di alcune particelle note e spin}
\label{tab_car_sim}
\end{table}
Le particelle che ubbidiscono al Principio di esclusione di Pauli e hanno carattere antisimmetrico sono dette \textit{fermioni}, hanno spin semi-intero $(1/2)$, mentre le particelle che non ubbidiscono al Principio di esclusione di Pauli e hanno carattere simmetrico sono dette \textit{bosoni}, hanno spin intero $(0, 1)$.
Il nome "fermione"/"bosone" deriva dal carattere di simmetria della particella.
Dunque i \textit{fermioni} sono descritti da funzioni d'onda antisimmetriche e i \textit{bosoni} sono descritti da funzioni d'onda simmetriche.

Abbiamo visto sopra che supponendo di avere due particelle che ubbidiscono al Principio di esclusione di Pauli (fermioni) e che si trovano nello stesso stato quantico $\alpha$ si ottiene una densità di probabilità di localizzarle nulla, hanno infatti carattere antisimmetrico.
Al contrario, nel caso dei bosoni, particelle descritte da funzioni d'onda simmetriche, la presenza di una particella un un certo stato quantico aumenta la probabilità di trovare anche l'altra nello stesso stato quantico.
Partendo dalla prima della \ref{funz_onda_sim_asim}, scrivo la funzione d'onda per i bosoni e suppongo di avere entrambi nello stesso stato quantico
\begin{equation}
\begin{split}
\psi_s & = \frac{1}{\sqrt{2}} \Bigl[ \psi_\alpha(1) \psi_\beta(2) + \psi_\beta(1) \psi_\alpha(2) \Bigr] \\
& \mbox{imponendo }  \beta = \alpha \\
\psi_s & = \frac{1}{\sqrt{2}} \Bigl[ \psi_\beta(1) \psi_\beta(2) + \psi_\beta(1) \psi_\beta(2) \Bigr] 
= \frac{ 2}{\sqrt{2} } \psi_\beta(1) \psi_\beta(2) \\
& = \sqrt{2} \psi_\beta(1) \psi_\beta(2)
\end{split}
\end{equation}
per cui la densità di probabilità diventa
\begin{equation}
\begin{split}
\psi^{\ast}_s \psi_s = 2 \psi^{\ast}_\beta(1)\psi^{\ast}_\beta(2)\psi_\beta(1)\psi_\beta(2) \not = 0
\end{split}
\end{equation}
Quindi la probabilità di trovare due bosoni, considerando l'indistinguibilità, è doppia rispetto al caso in cui non ne tenga conto:
\begin{equation}
\begin{split}
\psi & = \psi_\alpha(1) \psi_\beta(2) \\
& \mbox{imponendo }  \beta = \alpha \\
\psi & = \psi_\alpha(1) \psi_\beta(2) \\
\psi^{\ast}_s \psi_s & = \psi^{\ast}_\beta(1)\psi^{\ast}_\beta(2)\psi_\beta(1)\psi_\beta(2)
\end{split}
\end{equation}

In riferimento al capitolo precedente:
la funzione densità degli stati per i fotoni viene moltiplicata per un fattore $2$ poiché il fotone è un bosone.




