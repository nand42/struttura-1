%%
%% Author: dariochinelli
%% 2021-1-21
%%

\section{Distribuzione di Maxwell-Boltzmann}


I moltiplicatori di Legrange $\alpha$ e $\beta$ sono legati al numero di particelle e all'energia.

$$ \mbox{Siano: }  n_i = \frac{N}{Z} g_i e^{ - \beta E_i }  \mbox{ , }  Z = \sum_i e^{ - \beta E_i } $$

Definisco la funzione $F(E)$, tale che

$$ < F > =  \frac{1}{N} \sum_i n_i F(E_i) = \frac{1}{Z} \sum_i g_i e^{- \beta E_i} F(E_i) $$

Ad esempio consideriamo un sistema con due stati possibili $E_{1,2} = \pm \varepsilon  \Rightarrow g_1 = g_2 = 1 $ 

$$ Z = e^{ - \beta E_1 } + e^{ - \beta E_2 } = e^{ \beta \varepsilon } + e^{ - \beta \varepsilon } = 2 \cosh ( \beta \varepsilon ) $$

$$  < E > = \frac{1}{Z} = \Bigl(  E_1  e^{ - \beta E_1 } + E_2 e^{ - \beta E_2 }  \Bigr)
= \frac{ \varepsilon e^{-\beta \varepsilon }   \varepsilon e^{\beta \varepsilon }}{2 \cosh (\beta \varepsilon )}  = -  \varepsilon \tanh (\beta \varepsilon ) $$

Poiché il termine $\beta E $ deve essere adimensionale allora $\beta$ deve avere le dimensioni di un inverso di un'energia.

$$  U = \frac{N}{Z} \Bigl( g_1 E_1 e^{ - \beta E_1 } + g_2 E_2 e^{ - \beta E_2 }  + ... \Bigr)  
= \frac{N}{Z} \sum_i g_i E_i e^{- \beta E_i} = - \frac{N}{Z}  \frac{d}{d\beta} \Bigl( \sum_i g_i e^{- \beta E_i}  \Bigr) = $$

$$ = - \frac{N}{Z} \frac{d Z}{d \beta} = - N \frac{d}{d \beta} (\ln Z) \Rightarrow < E > = \frac{U}{N} = - \frac{d}{d \beta} ( \ln Z ) $$ 

Sia l'energia totale del sistema di particelle sia quella media dipendono da $\beta$.

Definisco ora la temperatura assoluta di sistemi in \textit{ equilibrio statistico } come $\beta = \frac{1}{k T} $ \\
dove $k = \SI{1.38e-23}{J/K} = \SI{8.62e-5}{eV/K}$

\begin{gather*} 
Z = \sum_i g_i e^{- \frac{E_i}{k T}}  \\
n_i = \frac{N}{Z} g_i e^{- \frac{E_i}{k T}}  \\
d\beta =  - \frac{d T}{k T^2} \\
U = k N T^2 \frac{d}{d T}(\ln Z)
\end{gather*}

$$ <E> = \frac{U}{N} = k T^2 \frac{d}{d T} (\ln Z) \Rightarrow <F> = \frac{1}{Z} \sum_i g_i F(E_i) e^{- \frac{E_i}{k T}} $$

Al crescere del rapporto $\frac{E_i}{k T}$ diminuisce $n_i$, il numero di occupazione egli stati.
Se fisso una temperatura $T$, al crescere di $E_i$ cala $n_i$,
dunque si ha probabilità di occupazione maggiore per gli stati ad energia $E$ minore ed anche a temperature minori.
Da notare è che allo zero assoluto si avrebbe un sovraffollamento totale.

$$ \frac{n_i}{n_j} = \frac{g_i}{g_j} \Bigl[ e^{- \frac{E_i - E_j}{k T}}  \Bigr] $$
Si possono confrontare $n_i$ ed $n_j$ con $ (E_i E_j ) = \Delta E \ll k T$

I gas ideali, sistemi di particelle identiche non interagenti considerando poche moli in un gran volume.

Consideriamo un numero di particelle comprese nell'intervallo energetico $E$ e $E+dE$, cioè si considera la Funzione densità degli stati:
$$ n(E)dE = \frac{N g(E) e^{-\beta E} dE}{\int_0^\infty g(E) e^{-\beta E} dE } $$
si può definire a partire da questo approccio classico e non solo, come visto in precedenza nell'equazione di Schrodinger.

L'energia di una particella libera è:
\begin{equation}
E = K = \frac{p^2}{2m}
\mbox{ con }
\begin{cases}
p = (p_x, p_y, p_z) \\
q = (x, y, z)
\end{cases}
\end{equation}

Si postula quindi che il numero di stati di energia disponibili sia proporzionale, e finito, rispetto al volume nello spazio delle fasi.

\begin{gather*} 
 N(E) \propto \int d^3 p d^3 q = 4 \pi V \int p^2 dp \\
\mbox{pongo }  x = \frac{p}{\sqrt{2 m E}} \Rightarrow x^2 \leq 1 \\
N(E) \propto 4 \pi V \int_{x^2 \le 1}^{x > 0} x^2 (2mE)^{\frac{3}{2}} dx = 4 \pi V \Bigl[ \frac{x^3}{3}  \Bigr]_0^1 (2mE)^{\frac{3}{2}} = \frac{4}{3} \pi V (2m)^{\frac{3}{2}} E^{\frac{3}{2}}
\end{gather*}

$$  g(E)dE = \frac{dN(E)}{dE}dE \Rightarrow g(E)dE \propto 4 \pi V (2m^3)^{\frac{1}{2}} E^{\frac{1}{2}} dE $$

Nello spazio delle fasi ogni stato di energia non è rappresentato da un punto ma da un volume $V = h^3$, per Heisemberg $\Delta x \Delta p \simeq h$.
Dunque la correzione quantistica alla fisica classica consiste nello scrivere che:


\begin{equation}\label{ge_de}
g(E) dE = \frac{4 \pi V (2m^3)^{\frac{1}{2}} E^{\frac{1}{2}} dE}{h^3}
\end{equation}


Si considera ad esempio l'oscillatore armonico unidimensionale $ E = \frac{p^2}{2m} + \frac{K q^2}{2} $

\begin{gather*} 
N(E) \propto \int dp dq \\
\frac{p^2}{2mE} + \frac{K q^2}{2E} \leq 1 \mbox{ sostituendo: } 
\begin{cases}
	x = \frac{p}{\sqrt{2mE}} \\
	y = \frac{q}{\sqrt{2E}} \sqrt{K}
\end{cases} \\
N(E) \propto \int_{x^2 + y^2 = 1} dx dy \frac{2E}{\sqrt{K}} \sqrt{m}  \mbox{ con } \omega = 2 \pi \nu = \sqrt{\frac{K}{m}} \\
N(E) \propto \frac{E}{\nu} \Rightarrow g(E)dE \propto \frac{dE}{\nu}
\end{gather*} 

$$ \bar\varepsilon =  \frac{1}{N} \int_0^{\infty} n(\varepsilon) \varepsilon d\varepsilon = \frac{N}{N} \frac{\int_0^{\infty} g(\varepsilon) e^{- \beta E} \varepsilon dE}{\int_0^{\infty} g(\varepsilon) e^{- \beta \varepsilon} \varepsilon dE }$$

È la stessa formula usata per il corpo nero, in cui $P(\varepsilon) = C e^{- \frac{\varepsilon}{k T}} $ e $ C = \frac{1}{k T}$ valida se il numero di stati è indipendente dall'energia. Per il corpo nero $g(\varepsilon)$ è costante.

$$ \bar\varepsilon =   \frac{\int_0^{\infty} \varepsilon p(\varepsilon) d \varepsilon}{\int_0^{\infty} p(\varepsilon) d \varepsilon}$$

\begin{gather*} \label{fun_partizione}
g(E)dE = \frac{4 \pi V (2m^3)^\frac{1}{2}  E^\frac{1}{2} }{ h^3 } \\
Z =  \frac{4 \pi V (2m^3)^\frac{1}{2} }{ h^3 }  \int_0^{\infty}  E^\frac{1}{2} e^{- \frac{E}{k T}}  dE \\
\mbox{pongo: } \beta = \frac{1}{k T} \mbox{ \space ;  }  E = x^2  \rightarrow dE = 2x dx \\
Z =  \frac{4 \pi V (2m^3)^\frac{1}{2} }{ h^3 }  \int_0^{\infty}  E^\frac{1}{2} e^{- \frac{E}{k T}}  dE \\
= \frac{4 \pi V (2m^3)^\frac{1}{2} }{ h^3 } \int_0^{\infty} 2x^2 e^{- \beta x^2 } dx \\
= \frac{8 \pi V (2m^3)^\frac{1}{2} }{ h^3 } \Bigl[  \frac{1}{4} \sqrt{\frac{\pi}{\beta^3}}  \Bigr] 
= \frac{V (2 \pi m k T)^\frac{3}{2}}{h^3}
\end{gather*}

Abbiamo ottenuto la \underline{funzione di partizione per un gas monoatomico ideale}.
Applicando a destra e sinistra il logaritmo naturale ottengo:

$$ \ln (Z) = C + \frac{3}{2} \ln ( k T ) $$
$$ <E> = k T^2 \frac{d}{dT} (\ln Z)  \Rightarrow <E> = \frac{3}{2} k T$$
Che è l'energia cinetica media di Boltzmann, da cui si deduce che $\beta = (k T)^{-1}$.

$$  U_{tot} =  \frac{3}{2} k N T = \frac{3}{2} n k N_A T = \frac{3}{2} n R T $$

dove si vede $R = k N_A = \SI{8.3145}{J / mol * k}$

\begin{equation} 
n(E) dE = d n = \frac{N}{Z} e^{- \frac{E}{k T} } g(E) dE = \frac{N}{Z} \frac{4 \pi V (2m^3)^\frac{1}{2} E^\frac{1}{2} }{ h^3 } e^{- \frac{E}{k T}}  dE 
\end{equation}


\begin{equation} \label{formula_maxwell}
\frac{dn}{dE} = \frac{2 \pi N}{(\pi k T)^\frac{3}{2}} E^\frac{1}{2} e^{-\frac{E}{k T}}
\end{equation}

Abbiamo ottenuto così la \underline{formula di Maxwell}.

Infine dall'equazione dell'energia cinetica $E = \frac{1}{2} m v^2$

\begin{equation}
\frac{dn}{dv} = \frac{dn}{dE} \frac{dE}{dv} = mv \frac{dn}{dE} = 4 \pi N (\frac{m}{2 \pi k T})^\frac{3}{2} v^2 e^{- \frac{m v^2}{2 k T}}
\end{equation}

La formula $<E> = \frac{3}{2} k T $ è l'energia cinetica per particella e deriva dal \\
\underline{Teorema di Equipartizione dell'energia}, 
che afferma che ogni particella ha energia cinetica media pari a $\frac{1}{2} k T$ per ogni grado di libertà posseduto.


















