%%
%% Author: dariochinelli
%% 2021-1-22
%%

\section{Degenerazione}

Riprendiamo le tre distribuzioni viste sopra:

\begin{equation}
\begin{split}
\mbox{\underline{Maxwell-Boltzmann}}  \quad\quad  \frac{n_s}{g_s} & = \frac{1}{e^{\alpha + \beta E_s}} \\
\mbox{\underline{Bose-Einstein}}  \quad\quad  \frac{n_s}{g_s} & = \frac{1}{e^{\alpha + \beta E_s} - 1} \\
\mbox{\underline{Fermi-Dirac}}  \quad\quad  \frac{n_s}{g_s} & = \frac{1}{e^{\alpha + \beta E_s} + 1 } 
\end{split}
\end{equation}

Si nota subito che se $e{\alpha + \beta E_s} \ll 1 $ si può comunque applicare la statistica classica, ma questo implica che allora  $\frac{n_s}{g_s} \ll 1 $,
situazione che si ha nel caso in cui il numero di particelle per stato sia molto minore di 1.
Se si prende come zero il livello più basso dell'energia $E_s = 0$, allora si ha che $e^{\beta E_s } \ge 1 \Rightarrow e^{\alpha} \gg 1 \Rightarrow e^{-\alpha} \ll 1$
dove $e^{-\alpha} = \xi $ è detto parametro di degenerazione.
Se questa condizione è soddisfatta e vale dunque la statistica classica, si dice che il gas considerato è \underline{non degenere}. 

\begin{equation}
\begin{split}
e^{\alpha} = \frac{Z}{N} \quad\quad Z & = \int_0^{\infty} g(E) e^{-\beta E} dE \\
& = \frac{4 \pi V (2m^3)^\frac{1}{2}}{h^3} \int_0^{\infty} E^{\frac{1}{2}} e^{-\frac{E}{kT}} dE
\end{split}
\end{equation}

\begin{equation}
\begin{split}
e^{-\alpha} \ll 1 \quad \Rightarrow \quad e^{-\alpha} = \frac{N}{Z} = \frac{N}{V} \frac{h^3}{(2 \pi m k T)^{\frac{3}{2}}} \ll 1 \\
\mbox{condizione verificata in due casi} \quad
\begin{cases} 
	\mbox{I.  Densità piccole} \\
	\mbox{II. Temperature alte}
\end{cases}
\end{split}
\end{equation}
E in tal caso è lecito applicare Maxwell-Boltzmann.

Consideriamo ora $Z = \frac{V}{\lambda_{TH}^3}$ dove $\lambda_{TH} = \Bigl(  \frac{h^2}{2 \pi m k T}  \Bigr)^{\frac{1}{2}}$ è la lunghezza termica di De Broglie.

Consideriamo un insieme di Bosoni, e calcoliamo il numero totale di particelle

\begin{equation}
N = \int_0^{\infty} n(E) dE = \int_0^{\infty} \frac{g(E) dE}{e^{\alpha} e^{\frac{E}{kT}} - 1 } 
\quad \mbox{dove} \quad 
g(E) dE = \frac{4 \pi V (2m^3)^{\frac{1}{2}}}{h^3} E^{\frac{1}{2}} dE
\end{equation}

Se pongo $x = \frac{E}{kT}$ e sapendo che $Z = \frac{(2 \pi m k T)^{\frac{3}{2}} V}{h^3}$ si ottiene $N = \frac{2 Z}{\sqrt{\pi}} \int_0^{\infty} \frac{\sqrt{x}}{e^{\alpha + x} - 1} dx$

Secondo la statistica di Bose-Einstein, si deve avere $\alpha > 0$ o si avrebbe un numero negativo di particelle per stato. 
Risolviamo ora l'integrale:

Dalla nota $\frac{1}{1 - x} = 1 + x + x^2 + ...$ si ha $(e^{\alpha x} - 1)^{-1} = \frac{e^{-\alpha - x}}{1 - e^{- \alpha - x} } = e^{- \alpha} (e^{-x} + e^{-\alpha - 2x} + ...)$

Dunque:
\begin{equation}
\begin{split}
N & = \frac{2 Z}{\sqrt{\pi}} \int_0^{\infty} \frac{\sqrt{x}}{e^{\alpha + x}} (1 - e^{-\alpha -x})^{-1} dx \\
& = \frac{2 Z}{\sqrt{\pi}} \int_0^{\infty} x^{\frac{1}{2}} e^{- \alpha -x } ( 1 + e^{-(\alpha - x)} + e^{- 2(\alpha + x)} + ...) dx \\
& = Z e^{-\alpha} (1 + \frac{1}{2^{\frac{3}{2}}} e^{- \alpha} + \frac{1}{3^{\frac{3}{2}}} e^{-2\alpha} + ... ) \\
& = Z e^{- \alpha} ( \sum_{j=1}^{+\infty} \frac{e^{- j \alpha}}{ (j + 1)^\frac{3}{2}} + 1) 
\end{split}
\end{equation}

Dove approssimando al primo ordine si ottiene proprio $ e^{ -\alpha } = \frac{ N }{ Z } $, calcoliamo ora l'energia totale:

\begin{equation}
\begin{split}
U & = \int_0^{\infty} n(E) E dE = \Bigl[  \mbox{ponendo } x = \frac{E}{kT}  \Bigr] \\
& = \frac{ 2 Z k T}{\sqrt{\pi} } \int_0^{\infty} \frac{ x^{ \frac{ 3}{2 } }}{e^{ \alpha + x } - 1 } dx \\
& = \frac{ 3}{2 } k T Z e^{ -\alpha } \Bigl(  1 + \frac{ e^{ -\alpha }}{2^{ \frac{ 5}{2 } } } + \frac{ e^{ - 2 \alpha }}{3^{ \frac{ 5}{2 } } }  + ... \Bigr) \\
& = \frac{ 3}{2 } k T Z e^{ -\alpha } \Bigl[ 1 + \sum_{j=1}^{ +\infty } \frac{ e^{ -j \alpha }}{(j + 1)^{ \frac{ 5}{2 } } } \Bigr]
\end{split}
\end{equation}

Approssimando al primo ordine si ottiene 
\begin{equation}
U = \frac{ 3}{2 } k T Z e^{ - \alpha} = \frac{ 3}{2 } N k T
\end{equation},
dunque l'energia media è data da:

\begin{equation}
E = \bar U = \frac{ U}{N } = \frac{ 3}{2 } k T \Bigl[ 1 - \frac{ 1}{2^{ \frac{ 5}{2 }} } \frac{ N h^3}{V (2 \pi m k T)^{ \frac{ 3}{2 }} } + ... \Bigr]
\end{equation}

Dove questo $V (2 \pi m k T)^{ \frac{ 3}{2 }} $ è il termine di DERIVAZIONE/???? del gas di bosoni rispetto al caso classico.
Se esso è $\ll 1$ allora il gas è degenere, altrimenti si deve applicare la statistica di Bose-Einstein.

I bosoni tendono a disporsi negli stati aventi energia minore rispetto a quanto accade nel gas classico, 
questo è il motivo per cui è ragionevole che $E = \bar U$ sia minore rispetto al caso classico.
Se considero invece un gas di fermioni basta rifare i conti sostituendo (+1) a (-1), per ottenere che:

\begin{equation}
<E> = \bar U = \frac{ 3}{2 } k T \Bigl[  1 + \frac{ 1}{2^{ \frac{ 5}{2 }}} \frac{ N h^3}{ V (2 \pi m k T)^{ \frac{ 3}{2 }} } + ...  \Bigr]
\end{equation}

In questo caso l'energia è maggiore rispetto al caso classico, e il motivo è da ricondurre al fatto che i fermioni sono soggetti al principio di Esclusione di Pauli;
per cui, pur di non occupare lo stesso stato, occupano anche gli stati aventi energia maggiore.
Una conseguenza del termine di degenerazione è che un gas di fermioni ha pressione maggiore, 
mentre uno di bosoni ha pressione minore, rispetto ad un gas classico.

Le statistiche quantistiche vengono applicate se le particelle sono indistinguibili, e questo si ha se la distanza fra le particelle è confrontabile con la loro $\lambda$ di De Broglie $(\lambda = \frac{ h}{p })$.

$$ K = \frac{ 3}{2} k T \quad p= \sqrt{2mK} = \sqrt{3mkT} \quad \lambda = \frac{ h}{(3mkT)^{ \frac{ 1}{2}}} $$


\textbf{Esempio:} \textit{Particelle vicine}\\
Si consideri un gas di volume $V$ e contenente $N$ particelle, allora il volume per ogni particella è $\frac{ V}{N} = d^3$.
Dunque $d$ è la distanza media tra una particella e l'altra e se $\lambda \ge d$ si hanno effetti quantistici perché le particelle risultano indistinguibili.

\begin{equation}
\begin{split}
& \frac{ h}{(3mkT)^{ \frac{ 1}{2}}} \ge (\frac{ V}{N})^{ \frac{ 1}{3}} \quad \Rightarrow \quad \frac{ h^3}{(3mkT)^{ \frac{ 3}{2}}} \ge \frac{ V}{N} \\
& \frac{ N}{V} \frac{ h^3}{(3mkT)^{ \frac{ 3}{2}}} \ge 1 \\
& \mbox{dato che } \quad \lambda = \frac{ h}{(3mkT)^{ \frac{ 1}{2}}} \quad \mbox{ e } \quad \frac{ V}{N} = d^3 \\
& \mbox{si ritrova } \quad \frac{ \lambda^3}{d^3} \ge 1
\end{split}
\end{equation}


















