%%
%% Author: dariochinelli
%% 2021-03-31
%%

\section{Degenerazione statistica}
Riprendiamo le tre distribuzioni statistiche viste sopra
\begin{equation}
\begin{split}
\mbox{\underline{Maxwell-Boltzmann}}  \quad\quad  \frac{n_s}{g_s} & = \frac{1}{e^{\alpha + \beta E_s}} \\
\mbox{\underline{Bose-Einstein}}  \quad\quad  \frac{n_s}{g_s} & = \frac{1}{e^{\alpha + \beta E_s} - 1} \\
\mbox{\underline{Fermi-Dirac}}  \quad\quad  \frac{n_s}{g_s} & = \frac{1}{e^{\alpha + \beta E_s} + 1 } 
\end{split}
\end{equation}
Si nota subito che se 
\begin{equation}
e^{\alpha + \beta E_s} \gg 1
\label{condizione_e}
\end{equation}
allora si può trascurare l'$1$ a denominatore e applicare la statistica classica, questo implica
\begin{equation}
\frac{n_s}{g_s} \ll 1
\end{equation}
per cui la statistica di Maxwell Boltzmann si applica nel caso in cui il numero di particelle per stato sia molto minore di 1.
Se si prende come $zero$ il livello più basso dell'energia o \textit{livello fondamentale} $E_s = 0$, allora si ha che l'espressione \ref{condizione_e} è verificata per
\begin{equation}
e^{\alpha} \gg 1 \quad\Leftrightarrow\quad  e^{-\alpha} \ll 1
\end{equation}


\subsection{Parametro di degenerazione} Introduciamo il parametro di degenerazione $\xi$ tale che 
\begin{equation}
e^{-\alpha} = \xi 
\end{equation}
Se la condizione 
\begin{equation}
\xi = e^{-\alpha} \ll 1
\label{cond_deg}
\end{equation}
è verificata allora si dice che il gas è un \underline{gas non degenere} e posso utilizzare la statistica classica;
se al contrario la condizione \ref{cond_deg} non è soddisfatta allora il gas si definisce \underline{gas degenere} e dovrà essere descritto dalle statistiche quantistiche.

\paragraph{Esplicitiamo la condizione di degenerazione}
Riprendendo la relazione nel caso della statistica di MB
\begin{equation}
e^{\alpha} = \frac{Z}{N}
\end{equation}
e la funzione di partizione \ref{funzione_partizione_gasmono}
\begin{equation}
Z = \frac{V (2\pi m k_B T)^{ \frac{3}{2} }}{h^3}
\end{equation}
che si può riscrivere 
\begin{equation}
Z = \frac{V}{\lambda_{th}^3}
\end{equation}
in funzione del parametro $\lambda_{th}$ detto \textit{lambda termica di De Broglie}
\begin{equation}
\lambda_{th} = \Bigl(  \frac{h^2}{2 \pi m k_B T}  \Bigr)^{ \frac{1}{2} }
\label{lambda_th_debroglie}
\end{equation}
allora il parametro di degenerazione diventa
\begin{equation}
\begin{split}
& e^{\alpha} = \frac{V}{N} \frac{1}{\lambda_{th}^3} = \frac{1}{n} \frac{1}{\lambda_{th}^3} \\
\Rightarrow\quad & \xi = e^{-\alpha} = n \lambda_{th}^3
\label{disaccordo1}
\end{split}
\end{equation}
equivalente a
\begin{equation}
e^{-\alpha} = \frac{N}{Z} =  \frac{N h^3}{V (2 \pi m k T)^{\frac{3}{2}}} \ll 1
\end{equation}
La condizione \ref{cond_deg} si ottiene per
\begin{itemize}
\item piccole densità $\quad\Rightarrow\quad n$ piccolo 
\item alte temperature $\quad\Rightarrow\quad T$ grande $\quad\Rightarrow\quad \lambda_{th}$ piccolo
\end{itemize}


\subsection{Gas di bosoni} consideriamo un gas ideale di bosoni, quindi composto da particelle che seguono la statistica di Bose Einstein, situazione abbastanza comune poiché le molecole di un gas hanno in genere \textit{spin} intero $(0, 1, ...)$.
La legge di distribuzione di Bose Einstein, nel caso abbia un continuo di livelli di energia, la posso scrivere come
\begin{equation}
n(E)dE = g(E) \frac{1}{e^{ \alpha } e^{ \frac{E}{k_B T} } - 1 } dE
\end{equation}
in cui la \textit{funzione densità degli stati} è
\begin{equation}
g(E)dE = \frac{4\pi V (2m^3)^{ \frac{1}{2} }}{h^3} E^{\frac{1}{2} } dE
\label{dens_stati}
\end{equation}
allora il numero totale delle particelle nel sistema
\begin{equation}
N = \int_0^{\infty} n(E)dE =  \int_0^{\infty} \frac{g(E) dE}{e^{\alpha} e^{ \frac{E}{k_B T} } - 1} 
\end{equation}
risolviamo allora questo integrale, successivamente imponendo
\begin{equation}
x = \frac{E}{kT} \quad\quad Z = \frac{V (2 \pi m k_B T)^{ \frac{3}{2} } }{h^3}
\end{equation}
e trovo
\begin{equation}
\begin{split}
N & = \frac{4\pi V (2m^3)^{ \frac{1}{2} }}{h^3} \int_0^{\infty} \frac{E^{ \frac{1}{2} dE }}{e^{\alpha} e^{ \frac{E}{k_B T} } - 1}  \\
&= \frac{2Z}{\sqrt{\pi}} \int_0^{\infty} \frac{x^{ \frac{1}{2} }}{e^{ \alpha + x } - 1} dx
\label{integrale_num_part}
\end{split}
\end{equation}
\textbf{NB} per la statistica di BE il parametro $\alpha$ \underline{deve} essere positivo, per non avere il rapporto $\frac{n_s}{g_s}$ negativo.
Il denominatore nell'integrale precedente \ref{integrale_num_part} si può riscrivere come
\begin{equation}
(e^{ \alpha + x } - 1)^{-1} = e^{ - \alpha - x } (1 - e^{ - \alpha - x })^{ -1 }
\end{equation}
per cui l'integrale, in cui la parentesi risulta essere la serie geometrica che si può sviluppare in serie di Taylor
$$ \frac{1}{1-y} = 1 + y + y^2 + ... \quad\Rightarrow\quad \frac{1}{1 - e^{ -\alpha - x }} = 1 + e^{ - ( \alpha + x) } + e^{ - 2 ( \alpha + x) } + ... $$
l'integrale \ref{integrale_num_part} lo posso riscrivere e calcolare come segue 
\begin{equation}
\begin{split}
N & = \frac{2Z}{\sqrt{\pi}} \int_0^{\infty} \frac{x^{ \frac{1}{2} }}{e^{ \alpha + x }} \Bigl(  1 - e^{ - \alpha - x }  \Bigr)^{ -1 } dx \\ 
& = \frac{2Z}{\sqrt{\pi}} \int_0^{\infty} \frac{x^{ \frac{1}{2} }}{e^{ \alpha + x }} \Bigl(  1 + e^{ - (\alpha + x) } + e^{ - 2 (\alpha + x) } + ...   \Bigr)
dx \\ &= Z e^{ -\alpha } \Bigl(  1 + \frac{1}{2^{\frac{3}{2}}}e^{ -\alpha } + \frac{1}{3^{\frac{3}{2}}} e^{ -2\alpha } + ...  \Bigr)
\end{split}
\end{equation}
da cui allora ricavo
\begin{equation}
e^{ -\alpha } = \frac{N}{Z} \Bigl(  1 + \frac{1}{2^{\frac{3}{2}}}e^{ -\alpha } + \frac{1}{3^{\frac{3}{2}}} e^{ -2\alpha } + ...  \Bigr)^{ -1 }
\end{equation}
fermandomi al primo termine $\quad\Rightarrow\quad $ \textbf{risultato classico}
\begin{equation}
e^{ -\alpha } = \frac{N}{Z} \quad\Rightarrow\quad e^{ \alpha } = \frac{Z}{N}
\end{equation}
fermarsi al primo termine significa assumere che $e^{ -\alpha } \ll 1$, appunto. \\
Calcoliamo ora l'\textbf{energia totale del sistema}, calcolando un integrale analogo al precedente tranne che per la dipendenza dalla temperatura $T$ a numeratore e la variabile integranda $x^{ \frac{3}{2} }$ 
\begin{equation}
\begin{split}
U & = \int_0^{\infty} E n(E)dE = \frac{2Zk_B T}{\sqrt{\pi}} \int_0^{\infty} \frac{x^{ \frac{3}{2} }}{e^{ \alpha + x } - 1 } dx \\
& = \frac{3}{2} k_B T Z e^{ -\alpha } \Bigl(  1 + \frac{1}{2^{ \frac{2}{5} }} e^{ -\alpha } +  \frac{1}{3^{ \frac{2}{5} }} e^{ -2 \alpha } + ... \Bigr)
\end{split}
\end{equation}
fermandomi al primo ordine
\begin{equation}
\begin{split}
U & = \frac{3}{2} k_B T Z e^{ -\alpha } \\
& = \frac{3}{2} k_B T Z \frac{N}{Z} = \frac{3}{2} N k_B T
\end{split}
\end{equation}
ricordando che $e^{ -\alpha } = \frac{N}{Z}$, ritrovo allora l'energia totale di un gas classico. \\
Da cui ricavo inoltre l'energia media per particella nel risultato classico
\begin{equation}
\frac{U}{N} = \langle E \rangle = \frac{3}{2} k_B T
\end{equation}
Fermandomi invece al secondo ordine
\begin{equation}
\bar U = \frac{U}{N} = \frac{3}{2} k_B T \Bigl[  1 - \frac{1}{2^{ \frac{5}{2} }}  \frac{N h^3}{V (2\pi m k_B T)^{ \frac{3}{2} }}  \Bigr]
\end{equation}
il secondo termine in parentesi esprime la \textit{deviazione} di un \textbf{gas di bosoni} rispetto ad un gas classico, per cui
\begin{equation}
\xi = \frac{N h^3}{V (2\pi m k_B T)^{ \frac{3}{2} }}  \ll 1 
\end{equation}
è la condizione per la statistica classica, che è proprio la quantità $e^{-\alpha}$ ovvero $\xi$.
Ciò significa che nel caso quantistico di \textit{bosoni}, rispetto ad un caso classico, l'energia del sistema sarà un po' \textbf{inferiore}:
la probabilità di trovare due particelle in uno stesso stato di energia è più grande per un gas di bosoni che per un gas classico, \textit{"i bosoni preferiscono stare in uno stesso stato"}.
Da questo ragionamento discende inoltre che un gas di bosoni ha una \textit{pressione inferiore} rispetto al gas classico, dovuto al fatto che hanno una energia media per particella inferiore.


\subsection{Gas di fermioni} Rifacendo tutto lo stesso ragionamento utilizzando la statistica di Fermi Dirac, quindi per un \textbf{gas di fermioni}, trovo che l'energia media per ogni fermione è 
\begin{equation}
\bar U = \frac{U}{N} = \frac{3}{2} k_B T \Bigl[  1 + \frac{1}{2^{ \frac{5}{2} }} \frac{N h^3}{V (2\pi m k_B T)^{ \frac{3}{2} }} \Bigr]
\end{equation}
il secondo termine in parentesi esprime la \textit{deviazione} di un \textbf{gas di fermioni} rispetto ad un gas classico, per cui
\begin{equation}
\xi = \frac{N h^3}{V (2\pi m k_B T)^{ \frac{3}{2} }}  \ll 1 
\end{equation}
è la condizione per la statistica classica, che, anche qui, è proprio la quantità $e^{-\alpha}$ ovvero $\xi$.
Ciò significa che nel caso quantistico di \textit{fermioni}, rispetto ad un caso classico, l'energia del sistema sarà un po' \textbf{maggiore}:
la probabilità di trovare due particelle in uno stesso stato di energia è nulla per un gas di fermioni che per un gas classico, infatti obbediscono al Principio di Esclusione di Pauli, \textit{"i fermioni preferiscono stare da soli in uno stesso stato"}.
Da questo ragionamento discende inoltre che un gas di bosoni ha una \textit{pressione maggiore} rispetto al gas classico, dovuto al fatto che hanno una energia media per particella maggiore.

\paragraph{NB:}
\textit{Ogni volta che la distanza tra le particelle diventa confrontabile o più piccola della lunghezza d'onda di De Broglie a loro assegnata, ci aspettiamo di osservare effetti quantistici, caso in cui le funzioni d'onda delle particelle si sovrappongono e non posso più trattarle classicamente.} \\
Supponiamo di avere un gas all'equilibrio ad una certa temperatura $T$, l'energia cinetica media per particella ed il momento medio sono
\begin{equation}
K_{media} = \frac{3}{2} k_B T
\quad\quad\quad
p_{medio} = \sqrt{2mK} = \sqrt{3 m k_B T}
\end{equation}
La lambda di De Broglie è \textit{proporzionale} alla \ref{lambda_th_debroglie} lambda termica di De Broglie $\lambda_{th}$, per cui
\begin{equation}
\lambda = \frac{h}{p} 
= \frac{h}{(3mk_B T)^{ \frac{1}{2} }}
= \Bigl(  \frac{2\pi}{3}  \Bigr)^{ \frac{1}{2} }\Bigl(  \frac{h^2}{2\pi mk_B T}  \Bigr)^{ \frac{1}{2} }
= \lambda_{th} \sqrt{\frac{2\pi}{3}} 
\end{equation}
Supponiamo allora di avere un sistema di $N$ particelle in un volume $V$, per cui il \textit{volume medio} per ogni particella e quindi la \textit{distanza media} saranno rispettivamente
\begin{equation}
d^3 = \frac{V}{N}
\quad\quad\quad
d = \Bigl(   \frac{V}{N}  \Bigr)^{\frac{1}{3}}
\end{equation}
gli \textit{effetti quantistici} sono importanti se
\begin{equation}
\lambda \ge d
\end{equation}
da cui sostituendo si trova quanto segue, da $(\ast)$ si trovano due risultati in parallelo:
\begin{equation}
\begin{split}
\Bigl(  \frac{2\pi}{3}  \Bigr)^{ \frac{1}{2} } \Bigl(  \frac{h^2}{2\pi mk_B T}  \Bigr)^{ \frac{1}{2} }  & \ge \Bigl(   \frac{V}{N}  \Bigr)^{\frac{1}{3}} \\
\frac{N}{V} \Bigl(  \frac{h^2}{2\pi mk_B T}  \Bigr)^{ \frac{3}{2} } & \ge \Bigl(  \frac{3}{2\pi}  \Bigr)^{ \frac{3}{2} }  \\
(\ast) & \\
\xi \ge  \Bigl(  \frac{3}{2\pi}  \Bigr)^{ \frac{3}{2} } \approx 0.33
\quad\quad\quad & \quad\quad
\frac{N}{V} \lambda_{th}^3 = n\lambda_{th}^3 \ge  \Bigl(  \frac{3}{2\pi}  \Bigr)^{ \frac{3}{2} } \approx 0.33
\end{split}
\end{equation}
in cui $\xi$ è il parametro di degenerazione
\begin{equation}
\xi = \frac{N h^3}{V(2\pi mk_B T)^{ \frac{3}{2} }}
\end{equation}
per cui si ottiene
\begin{equation}
\xi = \frac{N}{V} \lambda_{th}^3 = n \lambda_{th}^3
\end{equation}
Risultato in accordo con la \ref{disaccordo1}.

\textbf{Concludo} che il parametro di degenerazione deve essere maggiore di $\approx \frac{1}{3}$ affinché siano consistenti gli effetti quantistici, per cui il gas è degenere.
Intorno alla temperatura ambiente, per normali gas, si ha $\xi \ll 1$ e posso trattare il sistema con la statistica di Maxwell Boltzmann.
Se la temperatura è sufficientemente alta che la lambda termica di De Broglie $\lambda_{th}$ è molto più piccola della distanza fra particelle, allora un gas si comporta classicamente e posso applicare la statistica di Maxwell Boltzmann. 
Quando la massa $m$ diventa piccola o la temperatura $T$ diventa molto bassa o quando la densità $\frac{N}{V}$ diventa elevata dovrò considerare effetti quantistici.





