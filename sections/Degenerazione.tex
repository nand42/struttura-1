%%
%% Author: dariochinelli
%% 2021-03-28
%%

\section{Degenerazione statistica}
Riprendiamo le tre distribuzioni statistiche viste sopra
\begin{equation}
\begin{split}
\mbox{\underline{Maxwell-Boltzmann}}  \quad\quad  \frac{n_s}{g_s} & = \frac{1}{e^{\alpha + \beta E_s}} \\
\mbox{\underline{Bose-Einstein}}  \quad\quad  \frac{n_s}{g_s} & = \frac{1}{e^{\alpha + \beta E_s} - 1} \\
\mbox{\underline{Fermi-Dirac}}  \quad\quad  \frac{n_s}{g_s} & = \frac{1}{e^{\alpha + \beta E_s} + 1 } 
\end{split}
\end{equation}
Si nota subito che se 
\begin{equation}
e{\alpha + \beta E_s} \gg 1
\label{condizione_e}
\end{equation}
allora si può trascurare l'$1$ a denominatore e applicare la statistica classica, questo implica
\begin{equation}
\frac{n_s}{g_s} \ll 1
\end{equation}
per cui la statistica di Maxwell Boltzmann si applica nel caso in cui il numero di particelle per stato sia molto minore di 1.
Se si prende come $zero$ il livello più basso dell'energia o \textit{livello fondamentale} $E_s = 0$, allora si ha che l'espressione \ref{condizione_e} è verificata per
\begin{equation}
e^{\alpha} \gg 1 \quad\Leftrightarrow\quad  e^{-\alpha} \ll 1
\end{equation}

\paragraph{Parametro di degenerazione} introduciamo il parametro $\xi$ tale che 
\begin{equation}
e^{-\alpha} = \xi 
\end{equation}
Se la condizione 
\begin{equation}
\xi = e^{-\alpha} \ll 1
\label{cond_deg}
\end{equation}
è verificata allora si dice che il gas è un \underline{gas non degenere} e posso utilizzare la statistica classica;
se al contrario la condizione \ref{cond_deg} non è soddisfatta allora il gas si definisce \underline{gas degenere} e dovrà essere descritto dalle statistiche quantistiche.

\paragraph{Esplicitiamo la condizione di degenerazione}
Riprendendo la relazione nel caso della statistica di MB
\begin{equation}
e^{\alpha} = \frac{Z}{N}
\end{equation}
e la funzione di partizione \ref{funzione_partizione_gasmono}
\begin{equation}
Z = \frac{V (2\pi m k_B T)^{ \frac{3}{2} }}{h^3}
\end{equation}
che si può riscrivere 
\begin{equation}
Z = \frac{V}{\lambda_{th}^3}
\end{equation}
in funzione del parametro detto \textit{$\lambda$ termica di De Broglie}
\begin{equation}
\lambda_{th} = \Bigl(  \frac{h^2}{2 m k_B T}  \Bigr)^{ \frac{1}{2} }
\label{lambda_th_debroglie}
\end{equation}
allora il parametro di degenerazione diventa
\begin{equation}
\begin{split}
& e^{\alpha} = \frac{V}{N} \frac{1}{\lambda_{th}^3} = \frac{1}{n} \frac{1}{\lambda_{th}^3} \\
\Rightarrow\quad & \xi = e^{-\alpha} = n \lambda_{th}^3
\end{split}
\end{equation}
La condizione \ref{cond_deg} si otterrà con
\begin{itemize}
\item piccole densità $\quad\Rightarrow\quad n$ piccolo 
\item alte temperature $\quad\Rightarrow\quad T$ grande $\quad\Rightarrow\quad \lambda_{th}$ piccolo
\end{itemize}
equivalente a
\begin{equation}
e^{-\alpha} = \frac{N}{Z} =  \frac{N h^3}{V (2 \pi m k T)^{\frac{3}{2}}} \ll 1
\end{equation}

\paragraph{Gas di bosoni} consideriamo un gas ideale di bosoni, quindi composto da particelle che seguono la statistica di Bose Einstein, situazione abbastanza comune poiché le molecole di un gas hanno in genere \textit{spin} intero $(0, 1, ...)$.
La legge di distribuzione di Bose Einstein, nel caso abbia un continuo di livelli di energia, la posso scrivere come
\begin{equation}
n(E)dE = g(E) \frac{1}{e^{ \alpha } e^{ \frac{E}{k_B T} } - 1 } dE
\end{equation}
allora il numero totale delle particelle del sistema
\begin{equation}
\begin{split}
N & = \int_0^{\infty} \frac{1}{e^{ \frac{E}{k_B T} } - 1} g(E) dE \\
g(E)dE & = \frac{4\pi V (2m^3)^{ \frac{1}{2} }}{h^3} E^{\frac{1}{2} } dE
\end{split}
\end{equation}
risolviamo allora questo integrale
\begin{equation}
\begin{split}
pongo & \quad\quad x = \frac{E}{kT} \\
uso & \quad\quad Z = \frac{V (2 \pi m k_B T)^{ \frac{3}{2} } }{h^3}
\end{split}
\end{equation}
e trovo
\begin{equation}
N = \frac{2Z}{\sqrt{\pi}} \int_0^{\infty} \frac{x^{ \frac{1}{2} }}{e^{ \alpha + x } - 1} dx
\label{integrale_num_part}
\end{equation}
notare che per la statistica di BE il parametro $\alpha$ \underline{deve} essere positivo, per non rischiare di avere il rapporto $\frac{n_s}{g_s}$ negativo.
Il denominatore nell'integrale precedente \ref{integrale_num_part} si può riscrivere come
\begin{equation}
(e^{ \alpha + x } - 1)^{-1} = e^{ - \alpha - x } (1 - e^{ - \alpha - x })^{ -1 }
\end{equation}
per cui l'integrale diventa
\begin{equation}
N = \frac{2Z}{\sqrt{\pi}} \int_0^{\infty} \frac{x^{ \frac{1}{2} }}{e^{ \alpha + x }} (1 - e^{ - \alpha - x })^{ -1 } dx
\end{equation}
dove la parentesi risulta essere la serie geometrica, che si può sviluppare in serie di Taylor
$$ \frac{1}{1-y} = 1 + y + y^2 + ... $$
e per cui posso scrivere l'integrando come
\begin{equation}
N = \frac{2Z}{\sqrt{\pi}} \int_0^{\infty} \frac{x^{ \frac{1}{2} }}{e^{ \alpha + x }} (1 + e^{ - (\alpha + x) } + e^{ - 2 (\alpha + x) } + ... ) dx
\end{equation}
integrando questa espressione termine a termine ottengo (è la funzione \textit{gamma})
\begin{equation}
N = Z e^{ -\alpha } \Bigl(  1 + \frac{1}{2^{\frac{3}{2}}}e^{ -\alpha } + \frac{1}{3^{\frac{3}{2}}} e^{ -2\alpha } + ...  \Bigr)
\end{equation}
da cui allora 
\begin{equation}
e^{ -\alpha } = \frac{N}{Z} \Bigl(  1 + \frac{1}{2^{\frac{3}{2}}}e^{ -\alpha } + \frac{1}{3^{\frac{3}{2}}} e^{ -2\alpha } + ...  \Bigr)^{ -1 }
\end{equation}
fermandomi al primo termine $\quad\Rightarrow\quad $ \textbf{risultato classico}
\begin{equation}
e^{ -\alpha } = \frac{N}{Z} \quad\Rightarrow\quad e^{ \alpha } = \frac{Z}{N}
\end{equation}
fermarsi al primo termine significa assumere che $e^{ -\alpha } \ll 1$, appunto.


\paragraph{Energia totale} Calcoliamo ora l'energia totale del sistema, calcolando un integrale analogo al precedente tranne che per la dipendenza dalla temperatura $T$ a numeratore e la variabile $x^{ \frac{3}{2} }$ sotto integrale
\begin{equation}
\begin{split}
U & = \int_0^{\infty} E n(E)dE = \frac{2Zk_B T}{\sqrt{\pi}} \int_0^{\infty} \frac{x^{ \frac{3}{2} }}{e^{ \alpha + x } - 1 } dx \\
& = \frac{3}{2} k_B T Z e^{ -\alpha } \Bigl(  1 + \frac{1}{2^{ \frac{2}{5} }} e^{ -\alpha } +  \frac{1}{3^{ \frac{2}{5} }} e^{ -2 \alpha } + ... \Bigr)
\end{split}
\end{equation}
fermandomi al primo termine e ricordando che $e^{ -\alpha } = \frac{N}{Z}$
\begin{equation}
\begin{split}
U & = \frac{3}{2} k_B T Z e^{ -\alpha } \\
& = \frac{3}{2} k_B T Z \frac{N}{Z} = \frac{3}{2} N k_B T
\end{split}
\end{equation}
ritrovo allora l'energia totale di un gas classico. \\
Da cui ricavo inoltre l'energia media per particella nel risultato classico
\begin{equation}
\frac{U}{N} = \langle E \rangle = \frac{3}{2} k_B T
\end{equation}

Fermandomi invece al secondo termine
\begin{equation}
\bar U = \frac{U}{N} = \frac{3}{2} k_B T \Bigl[  1 - \frac{1}{2^{ \frac{5}{2} }}  \frac{N h^3}{V (2\pi m k_B T)^{ \frac{3}{2} }}  \Bigr]
\end{equation}
il secondo termine in parentesi esprime la \textit{deviazione} di un gas di bosoni da un gas classico, per cui
\begin{equation}
\frac{N h^3}{V (2\pi m k_B T)^{ \frac{3}{2} }}  \ll 1 
\end{equation}
è la condizione per la statistica classica, che è proprio la quantità $e^{-\alpha}$ ovvero $\xi$.
Ciò significa che nel caso quantistico dei bosoni, rispetto ad un caso classico, l'energia del sistema sarà un po' inferiore:
la probabilità di trovare due particelle in uno stesso stato di energia è più grande per un gas di bosoni che per un gas classico, "i bosoni preferiscono stare in uno stesso stato".
Da questo ragionamento discende inoltre che un gas di bosoni ha una \textit{pressione} inferiore rispetto al gas classico, hanno infatti una energia media per particella inferiore.










\newpage

\begin{equation}
\begin{split}
e^{\alpha} = \frac{Z}{N} \quad\quad Z & = \int_0^{\infty} g(E) e^{-\beta E} dE \\
& = \frac{4 \pi V (2m^3)^\frac{1}{2}}{h^3} \int_0^{\infty} E^{\frac{1}{2}} e^{-\frac{E}{kT}} dE
\end{split}
\end{equation}

\begin{equation}
\begin{split}
e^{-\alpha} \ll 1 \quad \Rightarrow \quad e^{-\alpha} = \frac{N}{Z} = \frac{N}{V} \frac{h^3}{(2 \pi m k T)^{\frac{3}{2}}} \ll 1 \\
\mbox{condizione verificata in due casi} \quad
\begin{cases} 
	\mbox{I.  Densità piccole} \\
	\mbox{II. Temperature alte}
\end{cases}
\end{split}
\end{equation}
E in tal caso è lecito applicare Maxwell-Boltzmann.

Consideriamo ora $Z = \frac{V}{\lambda_{TH}^3}$ dove $\lambda_{TH} = \Bigl(  \frac{h^2}{2 \pi m k T}  \Bigr)^{\frac{1}{2}}$ è la lunghezza termica di De Broglie.

Consideriamo un insieme di Bosoni, e calcoliamo il numero totale di particelle

\begin{equation}
N = \int_0^{\infty} n(E) dE = \int_0^{\infty} \frac{g(E) dE}{e^{\alpha} e^{\frac{E}{kT}} - 1 } 
\quad \mbox{dove} \quad 
g(E) dE = \frac{4 \pi V (2m^3)^{\frac{1}{2}}}{h^3} E^{\frac{1}{2}} dE
\end{equation}

Se pongo $x = \frac{E}{kT}$ e sapendo che $Z = \frac{(2 \pi m k T)^{\frac{3}{2}} V}{h^3}$ si ottiene $N = \frac{2 Z}{\sqrt{\pi}} \int_0^{\infty} \frac{\sqrt{x}}{e^{\alpha + x} - 1} dx$

Secondo la statistica di Bose-Einstein, si deve avere $\alpha > 0$ o si avrebbe un numero negativo di particelle per stato. 
Risolviamo ora l'integrale:

Dalla nota $\frac{1}{1 - x} = 1 + x + x^2 + ...$ si ha $(e^{\alpha x} - 1)^{-1} = \frac{e^{-\alpha - x}}{1 - e^{- \alpha - x} } = e^{- \alpha} (e^{-x} + e^{-\alpha - 2x} + ...)$

Dunque:
\begin{equation}
\begin{split}
N & = \frac{2 Z}{\sqrt{\pi}} \int_0^{\infty} \frac{\sqrt{x}}{e^{\alpha + x}} (1 - e^{-\alpha -x})^{-1} dx \\
& = \frac{2 Z}{\sqrt{\pi}} \int_0^{\infty} x^{\frac{1}{2}} e^{- \alpha -x } ( 1 + e^{-(\alpha - x)} + e^{- 2(\alpha + x)} + ...) dx \\
& = Z e^{-\alpha} (1 + \frac{1}{2^{\frac{3}{2}}} e^{- \alpha} + \frac{1}{3^{\frac{3}{2}}} e^{-2\alpha} + ... ) \\
& = Z e^{- \alpha} ( \sum_{j=1}^{+\infty} \frac{e^{- j \alpha}}{ (j + 1)^\frac{3}{2}} + 1) 
\end{split}
\end{equation}

Dove approssimando al primo ordine si ottiene proprio $ e^{ -\alpha } = \frac{ N }{ Z } $, calcoliamo ora l'energia totale:

\begin{equation}
\begin{split}
U & = \int_0^{\infty} n(E) E dE = \Bigl[  \mbox{ponendo } x = \frac{E}{kT}  \Bigr] \\
& = \frac{ 2 Z k T}{\sqrt{\pi} } \int_0^{\infty} \frac{ x^{ \frac{ 3}{2 } }}{e^{ \alpha + x } - 1 } dx \\
& = \frac{ 3}{2 } k T Z e^{ -\alpha } \Bigl(  1 + \frac{ e^{ -\alpha }}{2^{ \frac{ 5}{2 } } } + \frac{ e^{ - 2 \alpha }}{3^{ \frac{ 5}{2 } } }  + ... \Bigr) \\
& = \frac{ 3}{2 } k T Z e^{ -\alpha } \Bigl[ 1 + \sum_{j=1}^{ +\infty } \frac{ e^{ -j \alpha }}{(j + 1)^{ \frac{ 5}{2 } } } \Bigr]
\end{split}
\end{equation}

Approssimando al primo ordine si ottiene 
\begin{equation}
U = \frac{ 3}{2 } k T Z e^{ - \alpha} = \frac{ 3}{2 } N k T
\end{equation},
dunque l'energia media è data da:

\begin{equation}
E = \bar U = \frac{ U}{N } = \frac{ 3}{2 } k T \Bigl[ 1 - \frac{ 1}{2^{ \frac{ 5}{2 }} } \frac{ N h^3}{V (2 \pi m k T)^{ \frac{ 3}{2 }} } + ... \Bigr]
\end{equation}

Dove questo $V (2 \pi m k T)^{ \frac{ 3}{2 }} $ è il termine di DERIVAZIONE/???? del gas di bosoni rispetto al caso classico.
Se esso è $\ll 1$ allora il gas è degenere, altrimenti si deve applicare la statistica di Bose-Einstein.

I bosoni tendono a disporsi negli stati aventi energia minore rispetto a quanto accade nel gas classico, 
questo è il motivo per cui è ragionevole che $E = \bar U$ sia minore rispetto al caso classico.
Se considero invece un gas di fermioni basta rifare i conti sostituendo (+1) a (-1), per ottenere che:

\begin{equation}
<E> = \bar U = \frac{ 3}{2 } k T \Bigl[  1 + \frac{ 1}{2^{ \frac{ 5}{2 }}} \frac{ N h^3}{ V (2 \pi m k T)^{ \frac{ 3}{2 }} } + ...  \Bigr]
\end{equation}

In questo caso l'energia è maggiore rispetto al caso classico, e il motivo è da ricondurre al fatto che i fermioni sono soggetti al principio di Esclusione di Pauli;
per cui, pur di non occupare lo stesso stato, occupano anche gli stati aventi energia maggiore.
Una conseguenza del termine di degenerazione è che un gas di fermioni ha pressione maggiore, 
mentre uno di bosoni ha pressione minore, rispetto ad un gas classico.

Le statistiche quantistiche vengono applicate se le particelle sono indistinguibili, e questo si ha se la distanza fra le particelle è confrontabile con la loro $\lambda$ di De Broglie $(\lambda = \frac{ h}{p })$.

$$ K = \frac{ 3}{2} k T \quad p= \sqrt{2mK} = \sqrt{3mkT} \quad \lambda = \frac{ h}{(3mkT)^{ \frac{ 1}{2}}} $$


\textbf{Esempio:} \textit{Particelle vicine}\\
Si consideri un gas di volume $V$ e contenente $N$ particelle, allora il volume per ogni particella è $\frac{ V}{N} = d^3$.
Dunque $d$ è la distanza media tra una particella e l'altra e se $\lambda \ge d$ si hanno effetti quantistici perché le particelle risultano indistinguibili.

\begin{equation}
\begin{split}
& \frac{ h}{(3mkT)^{ \frac{ 1}{2}}} \ge (\frac{ V}{N})^{ \frac{ 1}{3}} \quad \Rightarrow \quad \frac{ h^3}{(3mkT)^{ \frac{ 3}{2}}} \ge \frac{ V}{N} \\
& \frac{ N}{V} \frac{ h^3}{(3mkT)^{ \frac{ 3}{2}}} \ge 1 \\
& \mbox{dato che } \quad \lambda = \frac{ h}{(3mkT)^{ \frac{ 1}{2}}} \quad \mbox{ e } \quad \frac{ V}{N} = d^3 \\
& \mbox{si ritrova } \quad \frac{ \lambda^3}{d^3} \ge 1
\end{split}
\end{equation}



